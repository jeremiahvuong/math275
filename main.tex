\documentclass{article}

\input{preamble}
\input{macros}
\input{letterfonts}

\title{\Huge{Math 275} Notes}
\author{\huge{Jeremiah Vuong}\\Los Angeles Valley College\\vuongjn5900@student.laccd.edu
}
\date{2024 Spring}

\begin{document}
\maketitle
\newpage

\pagebreak

\section{1}
\subsection{Slope (Direction) Fields}
Equations containing derivatives are differential equations. \\
A differential equation that describes some physical process is often called a mathematical model of the process.
\dfn{Slope (direction) field}{
  Given the differential equation $d y / d t=f(t, y)$. If we systematically evaluate $f$ over a rectangular grid of points in the ty-plane and draw a line element at each point $(t, y)$ of the grid with slope $f(t, y)$, then the collection of all these line elements is called a slope (direction) field of the differential equation $d y / d t=f(t, y)$
} \noindent
We can graph undefined slopes in a slope field by using a vertical line; horizontal for 0.
\qs{pg. 8: \#6}{
  Write down a differential equation of the form $\mathrm{dy} / \mathrm{dt}=\mathrm{ay}+\mathrm{b}$ whose solutions diverge from $\mathrm{y}=2$.
}
\sol \\
Given that the equation is dependent on just $y$, we can say that it is an autonomous differential equation (DE).
We can subsequently solve for the DE by setting $\frac{dy}{dt} = f(y) = 0$, such that its solution is its equilibrium solution.
Since $y=2$, $\boxed{\frac{dy}{dt} = y-2}$. Thus, any number that is less than 2 will result in the solution's phase portrait diverging (-) from $y=2$ --- vice versa for numbers greater than 2.
\subsection{tfi}
\subsection{Classification on DEs}
The order of a DE is the order of the highest derivative that appears in the equation.
\dfn{Linear ODE}{The ODE $F\left(t, y, y^{\prime}, \ldots, y^{(n)}\right)=0$ is said to be linear if $F$ is a linear function of the variables $y, y^{\prime}, \ldots, y^{(n)}$. Thus the general linear ODE of order $n$ is $a_0(t) y^{(n)}+a_1(t) y^{(n-1)}+\ldots+a_n(t) y=g(t)$.}
\qs{pg. 22}{
  Determine the order of the given DE; also state whether the equation is linear or nonlinear: \\
  1) $t^2 y^{\prime \prime}+t y^{\prime}+2 y=\sin t$ \\
  2) $\left(1+y^2\right) y^{\prime \prime}+t y^{\prime}+y=e^t$
}
\sol \\ 
1) The order of the DE is 2, and it is linear: noticed how all the coefficents of n derivatives of $y$ are functions of $t$. \\
2) The order of the DE is 2, and it is nonlinear: noticed how the coefficents of n derivatives of $y$ are functions of $t$ and $y$.

\dfn{Nth-order IDE solutions}{Any function $h$, defined on an interval and possessing at least $n$ derivatives that are continuous on this interval, which substituted into an nth-order ODE reduces the equation to an identity, is said to be a solution of the equation on the interval.}
\qs{pg. 22: \#9)}{
  Verify that the functions $y_1(t)=t^{-2}$ and $y_2(t)=t^{-2} \ln t$ are solutions of the DE
$$ t^2 y^{\prime \prime}+5 t y^{\prime}+4 y=0 $$
}
\sol \\
We can verify that $y_1(t)$ and $y_2(t)$ are solutions of the DE by substituting them into the DE and verifying that the equation holds true. \\
\begin{align*}
  y_1 &= t^{-2} \\
  y_1^{\prime} &= -2t^{-3} \\
  y_1^{\prime \prime} &= 6t^{-4} \\
\end{align*}
Substituting: $t^2 y^{\prime \prime}+5 t y^{\prime}+4 y  = t^2(6t^{-4})+5t(-2t^{-3})+4t^{-2} = 0$ \\
For homework 1, we can verify that $y_2(t)$ is a solution of the DE.

\qs{\#12}{
  Determine the values of $r$ for which $y^{\prime \prime}+y^{\prime}-6 y=0$ has solutions of the form $y=e^{r t}$
}
\sol \\
We can solve for the values of $r$ by substituting $y=e^{rt}$ into the DE and solving for $r$.
$y=e^{rt}$, $y^{\prime}=re^{rt}$, and $y^{\prime \prime}=r^2e^{rt}$. Substituting: $r^2e^{rt}+re^{rt}-6e^{rt}=0$.
Factoring out $e^{rt}$, we get $e^{rt}(r^2+r-6)=0$. Thus, $r^2+r-6=0$, and $\boxed{r=2, -3}$.

\qs{\#14} {
  Determine the values of $r$ for which $t^2 y^{\prime \prime}+4 t y^{\prime}+2 y=0$ has solutions of the form $y=t^r$ for $t>0$
}
\sol \\
$y=t^r$, $y^{\prime}=rt^{r-1}$, and $y^{\prime \prime}=r(r-1)t^{r-2}$.
Substituting: $t^2(r(r-1)t^{r-2})+4t(rt^{r-1})+2t^r=0$.
Factoring out $t^r$, we get $t^r[r(r-1)+4r+2]=0$. $t > 0$ such that $t^r \neq 0$, and $r(r-1)+4r+2=0$. Thus, $r^2+3r+2=0$, and $\boxed{r=-1, -2}$.
\section{tfi}
\subsection{Linear DEs}
\dfn{First-order linear DE}{
  An ODE of the form $d y / d t+p(t) y=g(t)$ is called a first-order linear differential equation in standard form
}
\ex{Solving a Linear First-Order DE}{
  i) Put the linear equation in standard form. \\
  ii) From the standard form of the equation identify $\mathrm{p}(\mathrm{t})$ and then find the integrating factor $e^{\int p(t) d t}$. No constant need be used in evaluating the indefinite integral in the exponent. \\
  iii) Multiply both sides of the standard form equation by the integrating factor. The left-hand side of the resulting equation is automatically the derivative of the product of the integrating factor and $\mathrm{y}: \quad \frac{d}{d t}\left[e^{\int p(t) d t} y\right]=e^{\int p(t) d t} g(t)$. \\
  iv) Integrate both sides of the last equation and solve for $y$
}
\subsection{Separable DEs}
\dfn{Seperable DE}{
  An ODE written in the differential form $M(x) d x+N(y) d y=0$ is said to be separable since terms involving each variable may be placed on opposite sides of the equation. A separable equation can be solved by integrating the functions $M$ and $N$.
}
\qs{pg. 38: \#4)}{
  Solve the DE:$\quad x y^{\prime}=\left(1-y^2\right)^{1 / 2}$
}
\sol \\
We can rewrite the DE as $x \frac{dy}{dx} = \sqrt{1-y^2}$,
we can then seperate the variables and integrate both sides.
$ x dy = \sqrt{1-y^2} dx \implies \int dy \frac{1}{\sqrt{1-y^2}} = \int dx \frac{1}{x}$.
Such that, $\arcsin(y) = \ln|x|+C$. Thus, $\boxed{y=\sin(\ln|x|+C)}$, we call this the general solution.

\dfn{General Order n-th ODE}{
  Trying to solve the ODE $\frac{d^n y}{d x^n}=f\left(x, y, y^{\prime}, \ldots, y^{(n-1)}\right)$ subject to the conditions $\quad y\left(x_0\right)=y_0, y^{\prime}\left(x_0\right)=y_1, \ldots, y^{(n-1)}\left(x_0\right)=y_{n-1}$, where $\mathrm{y}_0, \mathrm{y}_1, \ldots, \mathrm{y}_{\mathrm{n}-1}$ are arbitrary real constants, is called an nth-order initial-value problem $(I V P)$. The values of $\mathrm{y}(\mathrm{x})$ and its first $\mathrm{n}-1$ derivatives at $\mathrm{x}_0$ are called initial conditions (IC).
}

\qs{pg. 38: \#16)}{
  Find the solution of the initial value problem $\sin (2 x) d x+\cos (3 y) d y=0, \quad y\left(\frac{\pi}{2}\right)=\frac{\pi}{3}$ in explicit form.
}
\sol \\
We can solve the DE by seperating the variables and integrating both sides.
$\sin(2x)dx = -\cos(3y)dy \implies \int \sin(2x)dx = -\int \cos(3y)dy$.
Such that, $-\frac{1}{2}\cos(2x) = -\frac{1}{3}\sin(3y) + C$.
We can solve for $C$ by using the initial condition $y(\frac{\pi}{2}) = \frac{\pi}{3}$.
$-\frac{1}{2}\cos(\pi) = -\frac{1}{3}\sin(\pi) + C \implies C = \frac{1}{2}$.
Thus, $-\frac{1}{2} \cos(2x) = \frac{1}{3}\sin(3y) + \frac{1}{2}$.
\begin{align*}
  6[-\frac{1}{2} \cos(2x)] &= 6[-\frac{1}{3}\sin(3y) + \frac{1}{2}] \\
  -3\cos(2x) &= 2\sin(3y) + 3 \\
  -3-3cos(2x) &= -2\sin(3y) \\
  (3\cos(2x)+3) \cdot \frac{1}{2} &= \sin(3y) \\
  \Aboxed{y &= \frac{1}{3} \cdot \arcsin(\frac{3+3\cos(2x)}{2})}
\end{align*}

\dfn{Homogeneous DEs}{
  A first-order DE in differential form $M(x, y) d x+N(x, y) d y=0$ is said to be homogeneous if both coefficient functions $M$ and $N$ have the same degree.
}
\noindent
Either of the substitutions $y=u x$ or $x=v y$, where $u$ and $v$ are new dependent variables, will reduce a homogeneous equation to a separable first-order differential equation.
\double
Although either of the indicated substitutions can be used for every homogeneous differential equation, in practice we try $x=v y$ whenever the function $M$ is simpler than $N$, and $y=u x$ whenever $N$ is simpler than $M$.


\qs{pg. 39: \#26)}{
  Solve: $$\frac{d y}{d x}=\frac{x^2+x y+y^2}{x^2}$$
}
\sol \\
Seperating the differential, $x^2 dy = (x^2+xy+y^2)dx \implies 0 = (x^2 + xy+ y^2) dx - x^2 dy$.
Let $y=ux$, such that $dy = udx + xdu$. Where $0 = (x^2 + x(ux) + (ux)^2)dx - x^2(udx + xdu)$.
Distributing,
\begin{align*}
0 &= x^2 dx + ux^2 dx + u^2x^2 dx - ux^2dx - x^3 du \\
x^3 du &= x^2 dx + u^2x^2 dx \\
\int \frac{du}{u^2+1} &= \int \frac{dx}{x} \\
\arctan(u) &= \ln|x| + C \\
\arctan(\frac{y}{x}) &= \ln|x| + C \\
\Aboxed{y &= x\tan(\ln|x|+C)}
\end{align*}

\qs{\#28) }{
  Solve: $$\frac{d y}{d x}=\frac{4 y-3 x}{2 x-y}$$
}
\sol \\
Cross multplying, $(2x-y) dy = (4y-3x)dx \implies 0 = (4y-3x)dx + (y-2x)dy$.
Let $x=vy$, such that $dx = vdy + ydv$. Where $0 = (4y-3vy)(ydy+ydv) + (y-2vy)dy$.
Distributing,
\begin{align*}
  0 &= 4vydy+4y^2dv-3v^2ydy-3vy^2dv+ydy-2vydy \\
  0 &= 2vydy + 4y^2dv-3v^2ydy-3vy^2dv+ydy \\
  3vy^2dv-4y^2dv &= 2vydy-3vy^2dy-ydy \\
  y^2(3v-4)dv &= y(2v-3v^2+1)dy \\ 
  y^2(3v-4)dv &= -y(3v^2-2v-1) dy \\
  \int \frac{3v-4}{(v-1)(3v+1)} dv &= -\int \frac{1}{y} dy \\
\end{align*}
Breaking up the L.H.S. into partial fractions,
\begin{align*}
  \frac{3v-4}{(v-1)(3v+1)} &= \frac{A}{v-1} + \frac{B}{3v+1} \\
  3v-4 &= A(3v+1) + B(v-1) \\
\end{align*}
Let $v = -\frac{1}{3}$, such that $3(-\frac{1}{3})-4=A(0)+B(-\frac{1}{3}-1) \implies B = \frac{15}{4}$. \\
Let $v = 1$, such that $3(1)-4=A(3+1)+B(0) \implies A = -\frac{1}{4}$. \\
\begin{align*}
  \int \frac{-\frac{1}{4}}{v-1} + \frac{\frac{15}{4}}{3v+1} dv &= -\int \frac{1}{y} dy \\
  -\frac{1}{4} \ln|v-1| + \frac{15}{12} \ln|3v+1| &= -\ln|y| + C \\
  -\frac{1}{3} \ln|\frac{x}{y} - 1| + \frac{5}{4} \ln|\frac{3x}{y}+1| &= -\ln|y| + C \\
  \Aboxed{-\ln|\frac{x-y}{y}| + 5 \ln|\frac{3x+y}{y}| &= -4\ln|y| + C} \\
\end{align*}

\end{document}