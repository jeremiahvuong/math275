\documentclass{article}

%%%%%%%%%%%%%%%%%%%%%%%%%%%%%%%%%
% PACKAGE IMPORTS
%%%%%%%%%%%%%%%%%%%%%%%%%%%%%%%%%
\usepackage[tmargin=2cm,rmargin=1in,lmargin=1in,margin=0.85in,bmargin=2cm,footskip=.2in]{geometry}
\usepackage{amsmath,amsfonts,amsthm,amssymb,mathtools}
\usepackage[varbb]{newpxmath}
\usepackage{xfrac}
\usepackage[makeroom]{cancel}
\usepackage{mathtools}
\usepackage{bookmark}
\usepackage{enumitem}
\usepackage{hyperref,theoremref}
\hypersetup{
	pdftitle={Assignment},
	colorlinks=true, linkcolor=doc!90,
	bookmarksnumbered=true,
	bookmarksopen=true
}
\graphicspath{ {./images/} }
\usepackage[most,many,breakable]{tcolorbox}
\usepackage{xcolor}
\usepackage{varwidth}
\usepackage{varwidth}
\usepackage{etoolbox}
\usepackage{nameref}
\usepackage{multicol,array}
\usepackage{multirow}
\usepackage{tikz-cd}
\usepackage[ruled,vlined,linesnumbered]{algorithm2e}
\usepackage{comment} % enables the use of multi-line comments (\ifx \fi) 
\usepackage{import}
\usepackage{xifthen}
\usepackage{pdfpages}
\usepackage{transparent}
\usepackage{tikzsymbols}
\usepackage{fix-cm}

%%%%%%%%%%%%%%%%%%%%%%%%%%%%%%
% SELF MADE COLORS
%%%%%%%%%%%%%%%%%%%%%%%%%%%%%%

\definecolor{myg}{RGB}{56, 140, 70}
\definecolor{myb}{RGB}{45, 111, 177}
\definecolor{myr}{RGB}{199, 68, 64}
\definecolor{mytheorembg}{HTML}{F2F2F9}
\definecolor{mytheoremfr}{HTML}{00007B}
\definecolor{mylenmabg}{HTML}{FFFAF8}
\definecolor{mylenmafr}{HTML}{983b0f}
\definecolor{mypropbg}{HTML}{f2fbfc}
\definecolor{mypropfr}{HTML}{191971}
\definecolor{myexamplebg}{HTML}{F2FBF8}
\definecolor{myexamplefr}{HTML}{88D6D1}
\definecolor{myexampleti}{HTML}{2A7F7F}
\definecolor{mydefinitbg}{HTML}{E5E5FF}
\definecolor{mydefinitfr}{HTML}{3F3FA3}
\definecolor{notesgreen}{RGB}{0,162,0}
\definecolor{myp}{RGB}{197, 92, 212}
\definecolor{mygr}{HTML}{2C3338}
\definecolor{myred}{RGB}{127,0,0}
\definecolor{myyellow}{RGB}{169,121,69}
\definecolor{myexercisebg}{HTML}{F2FBF8}
\definecolor{myexercisefg}{HTML}{88D6D1}


\setlength{\parindent}{1cm}
%================================
% EXAMPLE BOX
%================================

\newtcbtheorem[number within=section]{Example}{Example}
{%
	colback = myexamplebg
	,breakable
	,colframe = myexamplefr
	,coltitle = myexampleti
	,boxrule = 1pt
	,sharp corners
	,detach title
	,before upper=\tcbtitle\par\smallskip
	,fonttitle = \bfseries
	,description font = \mdseries
	,separator sign none
	,description delimiters parenthesis
}
{ex}

%================================
% DEFINITION BOX
%================================

\newtcbtheorem[number within=section]{Definition}{Definition}{enhanced,
	before skip=2mm,after skip=2mm, colback=red!5,colframe=red!80!black,boxrule=0.5mm,
	attach boxed title to top left={xshift=1cm,yshift*=1mm-\tcboxedtitleheight}, varwidth boxed title*=-3cm,
	boxed title style={frame code={
					\path[fill=tcbcolback]
					([yshift=-1mm,xshift=-1mm]frame.north west)
					arc[start angle=0,end angle=180,radius=1mm]
					([yshift=-1mm,xshift=1mm]frame.north east)
					arc[start angle=180,end angle=0,radius=1mm];
					\path[left color=tcbcolback!60!black,right color=tcbcolback!60!black,
						middle color=tcbcolback!80!black]
					([xshift=-2mm]frame.north west) -- ([xshift=2mm]frame.north east)
					[rounded corners=1mm]-- ([xshift=1mm,yshift=-1mm]frame.north east)
					-- (frame.south east) -- (frame.south west)
					-- ([xshift=-1mm,yshift=-1mm]frame.north west)
					[sharp corners]-- cycle;
				},interior engine=empty,
		},
	fonttitle=\bfseries,
	title={#2},#1}{def}

%================================
% Solution BOX
%================================

\makeatletter
\newtcbtheorem[number within=section]{question}{Question}{enhanced,
	breakable,
	colback=white,
	colframe=myb!80!black,
	attach boxed title to top left={yshift*=-\tcboxedtitleheight},
	fonttitle=\bfseries,
	title={#2},
	boxed title size=title,
	boxed title style={%
			sharp corners,
			rounded corners=northwest,
			colback=tcbcolframe,
			boxrule=0pt,
		},
	underlay boxed title={%
			\path[fill=tcbcolframe] (title.south west)--(title.south east)
			to[out=0, in=180] ([xshift=5mm]title.east)--
			(title.center-|frame.east)
			[rounded corners=\kvtcb@arc] |-
			(frame.north) -| cycle;
		},
	#1
}{def}
\makeatother

%================================
% SOLUTION BOX
%================================

\makeatletter
\newtcolorbox{solution}{enhanced,
	breakable,
	colback=white,
	colframe=myg!80!black,
	attach boxed title to top left={yshift*=-\tcboxedtitleheight},
	title=Solution,
	boxed title size=title,
	boxed title style={%
			sharp corners,
			rounded corners=northwest,
			colback=tcbcolframe,
			boxrule=0pt,
		},
	underlay boxed title={%
			\path[fill=tcbcolframe] (title.south west)--(title.south east)
			to[out=0, in=180] ([xshift=5mm]title.east)--
			(title.center-|frame.east)
			[rounded corners=\kvtcb@arc] |-
			(frame.north) -| cycle;
		},
}
\makeatother

%================================
% Question BOX
%================================

\makeatletter
\newtcolorbox{qstion}[2][]{%
    enhanced,
    breakable,
    colback=white,
    colframe=mygr,
    attach boxed title to top left={yshift*=-\tcboxedtitleheight},
    fonttitle=\bfseries,
    title={#2},
    boxed title size=title,
    boxed title style={%
        sharp corners,
        rounded corners=northwest,
        colback=tcbcolframe,
        boxrule=0pt,
    },
    underlay boxed title={%
        \path[fill=tcbcolframe] (title.south west)--(title.south east)
        to[out=0, in=180] ([xshift=5mm]title.east)--
        (title.center-|frame.east)
        [rounded corners=\kvtcb@arc] |-
        (frame.north) -| cycle;
    },
    #1
}

\makeatother

\newtcbtheorem[number within=chapter]{wconc}{Wrong Concept}{
	breakable,
	enhanced,
	colback=white,
	colframe=myr,
	arc=0pt,
	outer arc=0pt,
	fonttitle=\bfseries\sffamily\large,
	colbacktitle=myr,
	attach boxed title to top left={},
	boxed title style={
			enhanced,
			skin=enhancedfirst jigsaw,
			arc=3pt,
			bottom=0pt,
			interior style={fill=myr}
		},
	#1
}{def}

%================================
% NOTE BOX
%================================

\usetikzlibrary{arrows,calc,shadows.blur}
\tcbuselibrary{skins}
\newtcolorbox{note}[1][]{%
	enhanced jigsaw,
	colback=gray!20!white,%
	colframe=gray!80!black,
	size=small,
	boxrule=1pt,
	title=\textbf{Note:-},
	halign title=flush center,
	coltitle=black,
	breakable,
	drop shadow=black!50!white,
	attach boxed title to top left={xshift=1cm,yshift=-\tcboxedtitleheight/2,yshifttext=-\tcboxedtitleheight/2},
	minipage boxed title=1.5cm,
	boxed title style={%
			colback=white,
			size=fbox,
			boxrule=1pt,
			boxsep=2pt,
			underlay={%
					\coordinate (dotA) at ($(interior.west) + (-0.5pt,0)$);
					\coordinate (dotB) at ($(interior.east) + (0.5pt,0)$);
					\begin{scope}
						\clip (interior.north west) rectangle ([xshift=3ex]interior.east);
						\filldraw [white, blur shadow={shadow opacity=60, shadow yshift=-.75ex}, rounded corners=2pt] (interior.north west) rectangle (interior.south east);
					\end{scope}
					\begin{scope}[gray!80!black]
						\fill (dotA) circle (2pt);
						\fill (dotB) circle (2pt);
					\end{scope}
				},
		},
	#1,
}

%================================
% THEOREM BOX
%================================

\tcbuselibrary{theorems,skins,hooks}
\newtcbtheorem[number within=section]{Theorem}{Theorem}
{%
	enhanced,
	breakable,
	colback = mytheorembg,
	frame hidden,
	boxrule = 0sp,
	borderline west = {2pt}{0pt}{mytheoremfr},
	sharp corners,
	detach title,
	before upper = \tcbtitle\par\smallskip,
	coltitle = mytheoremfr,
	fonttitle = \bfseries\sffamily,
	description font = \mdseries,
	separator sign none,
	segmentation style={solid, mytheoremfr},
}
{th}

%================================
% Custom
%================================

\newcommand{\ex}[2]{\begin{Example}{#1}{}#2\end{Example}}
\newcommand{\dfn}[2]{\begin{Definition}[colbacktitle=red!75!black]{#1}{}#2\end{Definition}}
\newcommand{\qs}[2]{\begin{question}{#1}{}#2\end{question}}
\newcommand{\hw}[2]{\begin{qstion}{#1}{}#2\end{qstion}}
\newcommand{\nt}[1]{\begin{note}#1\end{note}}
\newcommand{\thm}[2]{\begin{Theorem}{#1}{}#2\end{Theorem}}
\newcommand{\pf}[2]{\begin{myproof}[#1]#2\end{myproof}}

\newenvironment{myproof}[1][\proofname]{%
	\proof[\bfseries #1: ]%
}{\endproof}

\newcommand{\imgg}[2]{\begin{center}\includegraphics[scale=#2]{#1}\end{center}}
%From M275 "Topology" at SJSU
\newcommand{\id}{\mathrm{id}}
\newcommand{\taking}[1]{\xrightarrow{#1}}
\newcommand{\inv}{^{-1}}

%From M170 "Introduction to Graph Theory" at SJSU
\DeclareMathOperator{\diam}{diam}
\DeclareMathOperator{\ord}{ord}
\newcommand{\defeq}{\overset{\mathrm{def}}{=}}

%From the USAMO .tex files
\newcommand{\ts}{\textsuperscript}
\newcommand{\dg}{^\circ}
\newcommand{\ii}{\item}

% % From Math 55 and Math 145 at Harvard
% \newenvironment{subproof}[1][Proof]{%
% \begin{proof}[#1] \renewcommand{\qedsymbol}{$\blacksquare$}}%
% {\end{proof}}

\newcommand{\liff}{\leftrightarrow}
\newcommand{\lthen}{\rightarrow}
\newcommand{\opname}{\operatorname}
\newcommand{\surjto}{\twoheadrightarrow}
\newcommand{\injto}{\hookrightarrow}
\newcommand{\On}{\mathrm{On}} % ordinals
\DeclareMathOperator{\img}{im} % Image
\DeclareMathOperator{\Img}{Im} % Image
\DeclareMathOperator{\coker}{coker} % Cokernel
\DeclareMathOperator{\Coker}{Coker} % Cokernel
\DeclareMathOperator{\Ker}{Ker} % Kernel
\DeclareMathOperator{\rank}{rank}
\DeclareMathOperator{\Spec}{Spec} % spectrum
\DeclareMathOperator{\Tr}{Tr} % trace
\DeclareMathOperator{\pr}{pr} % projection
\DeclareMathOperator{\ext}{ext} % extension
\DeclareMathOperator{\pred}{pred} % predecessor
\DeclareMathOperator{\dom}{dom} % domain
\DeclareMathOperator{\ran}{ran} % range
\DeclareMathOperator{\Hom}{Hom} % homomorphism
\DeclareMathOperator{\Mor}{Mor} % morphisms
\DeclareMathOperator{\End}{End} % endomorphism

\newcommand{\eps}{\epsilon}
\newcommand{\veps}{\varepsilon}
\newcommand{\ol}{\overline}
\newcommand{\ul}{\underline}
\newcommand{\wt}{\widetilde}
\newcommand{\wh}{\widehat}
\newcommand{\vocab}[1]{\textbf{\color{blue} #1}}
\providecommand{\half}{\frac{1}{2}}
\newcommand{\dang}{\measuredangle} %% Directed angle
\newcommand{\ray}[1]{\overrightarrow{#1}}
\newcommand{\seg}[1]{\overline{#1}}
\newcommand{\arc}[1]{\wideparen{#1}}
\DeclareMathOperator{\cis}{cis}
\DeclareMathOperator*{\lcm}{lcm}
\DeclareMathOperator*{\argmin}{arg min}
\DeclareMathOperator*{\argmax}{arg max}
\newcommand{\cycsum}{\sum_{\mathrm{cyc}}}
\newcommand{\symsum}{\sum_{\mathrm{sym}}}
\newcommand{\cycprod}{\prod_{\mathrm{cyc}}}
\newcommand{\symprod}{\prod_{\mathrm{sym}}}
\newcommand{\Qed}{\begin{flushright}\qed\end{flushright}}
\newcommand{\parinn}{\setlength{\parindent}{1cm}}
\newcommand{\parinf}{\setlength{\parindent}{0cm}}
% \newcommand{\norm}{\|\cdot\|}
\newcommand{\inorm}{\norm_{\infty}}
\newcommand{\opensets}{\{V_{\alpha}\}_{\alpha\in I}}
\newcommand{\oset}{V_{\alpha}}
\newcommand{\opset}[1]{V_{\alpha_{#1}}}
\newcommand{\lub}{\text{lub}}
\newcommand{\del}[2]{\frac{\partial #1}{\partial #2}}
\newcommand{\Del}[3]{\frac{\partial^{#1} #2}{\partial^{#1} #3}}
\newcommand{\deld}[2]{\dfrac{\partial #1}{\partial #2}}
\newcommand{\Deld}[3]{\dfrac{\partial^{#1} #2}{\partial^{#1} #3}}
\newcommand{\lm}{\lambda}
\newcommand{\uin}{\mathbin{\rotatebox[origin=c]{90}{$\in$}}}
\newcommand{\usubset}{\mathbin{\rotatebox[origin=c]{90}{$\subset$}}}
\newcommand{\lt}{\left}
\newcommand{\rt}{\right}
\newcommand{\bs}[1]{\boldsymbol{#1}}
\newcommand{\exs}{\exists}
\newcommand{\st}{\strut}
\newcommand{\dps}[1]{\displaystyle{#1}}

\newcommand{\sol}{\setlength{\parindent}{0cm}\textbf{\textit{Solution:}}\setlength{\parindent}{1cm} }
\newcommand{\solve}[1]{\setlength{\parindent}{0cm}\textbf{\textit{Solution: }}\setlength{\parindent}{1cm}#1 \Qed}

\newcommand{\double}{\\\\}
\newcommand{\triple}{\\\\\\}

\newcommand{\nlist}[1]{\begin{enumerate}#1\end{enumerate}}
\newcommand{\nab}{\vec{\nabla}}
% Things Lie
\newcommand{\kb}{\mathfrak b}
\newcommand{\kg}{\mathfrak g}
\newcommand{\kh}{\mathfrak h}
\newcommand{\kn}{\mathfrak n}
\newcommand{\ku}{\mathfrak u}
\newcommand{\kz}{\mathfrak z}
\DeclareMathOperator{\Ext}{Ext} % Ext functor
\DeclareMathOperator{\Tor}{Tor} % Tor functor
\newcommand{\gl}{\opname{\mathfrak{gl}}} % frak gl group
\renewcommand{\sl}{\opname{\mathfrak{sl}}} % frak sl group chktex 6

% More script letters etc.
\newcommand{\SA}{\mathcal A}
\newcommand{\SB}{\mathcal B}
\newcommand{\SC}{\mathcal C}
\newcommand{\SF}{\mathcal F}
\newcommand{\SG}{\mathcal G}
\newcommand{\SH}{\mathcal H}
\newcommand{\OO}{\mathcal O}

\newcommand{\SCA}{\mathscr A}
\newcommand{\SCB}{\mathscr B}
\newcommand{\SCC}{\mathscr C}
\newcommand{\SCD}{\mathscr D}
\newcommand{\SCE}{\mathscr E}
\newcommand{\SCF}{\mathscr F}
\newcommand{\SCG}{\mathscr G}
\newcommand{\SCH}{\mathscr H}

% Mathfrak primes
\newcommand{\km}{\mathfrak m}
\newcommand{\kp}{\mathfrak p}
\newcommand{\kq}{\mathfrak q}

% number sets
\newcommand{\RR}[1][]{\ensuremath{\ifstrempty{#1}{\mathbb{R}}{\mathbb{R}^{#1}}}}
\newcommand{\NN}[1][]{\ensuremath{\ifstrempty{#1}{\mathbb{N}}{\mathbb{N}^{#1}}}}
\newcommand{\ZZ}[1][]{\ensuremath{\ifstrempty{#1}{\mathbb{Z}}{\mathbb{Z}^{#1}}}}
\newcommand{\QQ}[1][]{\ensuremath{\ifstrempty{#1}{\mathbb{Q}}{\mathbb{Q}^{#1}}}}
\newcommand{\CC}[1][]{\ensuremath{\ifstrempty{#1}{\mathbb{C}}{\mathbb{C}^{#1}}}}
\newcommand{\PP}[1][]{\ensuremath{\ifstrempty{#1}{\mathbb{P}}{\mathbb{P}^{#1}}}}
\newcommand{\HH}[1][]{\ensuremath{\ifstrempty{#1}{\mathbb{H}}{\mathbb{H}^{#1}}}}
\newcommand{\FF}[1][]{\ensuremath{\ifstrempty{#1}{\mathbb{F}}{\mathbb{F}^{#1}}}}
% expected value
\newcommand{\EE}{\ensuremath{\mathbb{E}}}
\newcommand{\charin}{\text{ char }}
\DeclareMathOperator{\sign}{sign}
\DeclareMathOperator{\Aut}{Aut}
\DeclareMathOperator{\Inn}{Inn}
\DeclareMathOperator{\Syl}{Syl}
\DeclareMathOperator{\Gal}{Gal}
\DeclareMathOperator{\GL}{GL} % General linear group
\DeclareMathOperator{\SL}{SL} % Special linear group

%---------------------------------------
% BlackBoard Math Fonts :-
%---------------------------------------

%Captital Letters
\newcommand{\bbA}{\mathbb{A}}	\newcommand{\bbB}{\mathbb{B}}
\newcommand{\bbC}{\mathbb{C}}	\newcommand{\bbD}{\mathbb{D}}
\newcommand{\bbE}{\mathbb{E}}	\newcommand{\bbF}{\mathbb{F}}
\newcommand{\bbG}{\mathbb{G}}	\newcommand{\bbH}{\mathbb{H}}
\newcommand{\bbI}{\mathbb{I}}	\newcommand{\bbJ}{\mathbb{J}}
\newcommand{\bbK}{\mathbb{K}}	\newcommand{\bbL}{\mathbb{L}}
\newcommand{\bbM}{\mathbb{M}}	\newcommand{\bbN}{\mathbb{N}}
\newcommand{\bbO}{\mathbb{O}}	\newcommand{\bbP}{\mathbb{P}}
\newcommand{\bbQ}{\mathbb{Q}}	\newcommand{\bbR}{\mathbb{R}}
\newcommand{\bbS}{\mathbb{S}}	\newcommand{\bbT}{\mathbb{T}}
\newcommand{\bbU}{\mathbb{U}}	\newcommand{\bbV}{\mathbb{V}}
\newcommand{\bbW}{\mathbb{W}}	\newcommand{\bbX}{\mathbb{X}}
\newcommand{\bbY}{\mathbb{Y}}	\newcommand{\bbZ}{\mathbb{Z}}

%---------------------------------------
% MathCal Fonts :-
%---------------------------------------

%Captital Letters
\newcommand{\mcA}{\mathcal{A}}	\newcommand{\mcB}{\mathcal{B}}
\newcommand{\mcC}{\mathcal{C}}	\newcommand{\mcD}{\mathcal{D}}
\newcommand{\mcE}{\mathcal{E}}	\newcommand{\mcF}{\mathcal{F}}
\newcommand{\mcG}{\mathcal{G}}	\newcommand{\mcH}{\mathcal{H}}
\newcommand{\mcI}{\mathcal{I}}	\newcommand{\mcJ}{\mathcal{J}}
\newcommand{\mcK}{\mathcal{K}}	\newcommand{\mcL}{\mathcal{L}}
\newcommand{\mcM}{\mathcal{M}}	\newcommand{\mcN}{\mathcal{N}}
\newcommand{\mcO}{\mathcal{O}}	\newcommand{\mcP}{\mathcal{P}}
\newcommand{\mcQ}{\mathcal{Q}}	\newcommand{\mcR}{\mathcal{R}}
\newcommand{\mcS}{\mathcal{S}}	\newcommand{\mcT}{\mathcal{T}}
\newcommand{\mcU}{\mathcal{U}}	\newcommand{\mcV}{\mathcal{V}}
\newcommand{\mcW}{\mathcal{W}}	\newcommand{\mcX}{\mathcal{X}}
\newcommand{\mcY}{\mathcal{Y}}	\newcommand{\mcZ}{\mathcal{Z}}


%---------------------------------------
% Bold Math Fonts :-
%---------------------------------------

%Captital Letters
\newcommand{\bmA}{\boldsymbol{A}}	\newcommand{\bmB}{\boldsymbol{B}}
\newcommand{\bmC}{\boldsymbol{C}}	\newcommand{\bmD}{\boldsymbol{D}}
\newcommand{\bmE}{\boldsymbol{E}}	\newcommand{\bmF}{\boldsymbol{F}}
\newcommand{\bmG}{\boldsymbol{G}}	\newcommand{\bmH}{\boldsymbol{H}}
\newcommand{\bmI}{\boldsymbol{I}}	\newcommand{\bmJ}{\boldsymbol{J}}
\newcommand{\bmK}{\boldsymbol{K}}	\newcommand{\bmL}{\boldsymbol{L}}
\newcommand{\bmM}{\boldsymbol{M}}	\newcommand{\bmN}{\boldsymbol{N}}
\newcommand{\bmO}{\boldsymbol{O}}	\newcommand{\bmP}{\boldsymbol{P}}
\newcommand{\bmQ}{\boldsymbol{Q}}	\newcommand{\bmR}{\boldsymbol{R}}
\newcommand{\bmS}{\boldsymbol{S}}	\newcommand{\bmT}{\boldsymbol{T}}
\newcommand{\bmU}{\boldsymbol{U}}	\newcommand{\bmV}{\boldsymbol{V}}
\newcommand{\bmW}{\boldsymbol{W}}	\newcommand{\bmX}{\boldsymbol{X}}
\newcommand{\bmY}{\boldsymbol{Y}}	\newcommand{\bmZ}{\boldsymbol{Z}}
%Small Letters
\newcommand{\bma}{\boldsymbol{a}}	\newcommand{\bmb}{\boldsymbol{b}}
\newcommand{\bmc}{\boldsymbol{c}}	\newcommand{\bmd}{\boldsymbol{d}}
\newcommand{\bme}{\boldsymbol{e}}	\newcommand{\bmf}{\boldsymbol{f}}
\newcommand{\bmg}{\boldsymbol{g}}	\newcommand{\bmh}{\boldsymbol{h}}
\newcommand{\bmi}{\boldsymbol{i}}	\newcommand{\bmj}{\boldsymbol{j}}
\newcommand{\bmk}{\boldsymbol{k}}	\newcommand{\bml}{\boldsymbol{l}}
\newcommand{\bmm}{\boldsymbol{m}}	\newcommand{\bmn}{\boldsymbol{n}}
\newcommand{\bmo}{\boldsymbol{o}}	\newcommand{\bmp}{\boldsymbol{p}}
\newcommand{\bmq}{\boldsymbol{q}}	\newcommand{\bmr}{\boldsymbol{r}}
\newcommand{\bms}{\boldsymbol{s}}	\newcommand{\bmt}{\boldsymbol{t}}
\newcommand{\bmu}{\boldsymbol{u}}	\newcommand{\bmv}{\boldsymbol{v}}
\newcommand{\bmw}{\boldsymbol{w}}	\newcommand{\bmx}{\boldsymbol{x}}
\newcommand{\bmy}{\boldsymbol{y}}	\newcommand{\bmz}{\boldsymbol{z}}

%---------------------------------------
% Scr Math Fonts :-
%---------------------------------------

\newcommand{\sA}{{\mathscr{A}}}   \newcommand{\sB}{{\mathscr{B}}}
\newcommand{\sC}{{\mathscr{C}}}   \newcommand{\sD}{{\mathscr{D}}}
\newcommand{\sE}{{\mathscr{E}}}   \newcommand{\sF}{{\mathscr{F}}}
\newcommand{\sG}{{\mathscr{G}}}   \newcommand{\sH}{{\mathscr{H}}}
\newcommand{\sI}{{\mathscr{I}}}   \newcommand{\sJ}{{\mathscr{J}}}
\newcommand{\sK}{{\mathscr{K}}}   \newcommand{\sL}{{\mathscr{L}}}
\newcommand{\sM}{{\mathscr{M}}}   \newcommand{\sN}{{\mathscr{N}}}
\newcommand{\sO}{{\mathscr{O}}}   \newcommand{\sP}{{\mathscr{P}}}
\newcommand{\sQ}{{\mathscr{Q}}}   \newcommand{\sR}{{\mathscr{R}}}
\newcommand{\sS}{{\mathscr{S}}}   \newcommand{\sT}{{\mathscr{T}}}
\newcommand{\sU}{{\mathscr{U}}}   \newcommand{\sV}{{\mathscr{V}}}
\newcommand{\sW}{{\mathscr{W}}}   \newcommand{\sX}{{\mathscr{X}}}
\newcommand{\sY}{{\mathscr{Y}}}   \newcommand{\sZ}{{\mathscr{Z}}}


%---------------------------------------
% Math Fraktur Font
%---------------------------------------

%Captital Letters
\newcommand{\mfA}{\mathfrak{A}}	\newcommand{\mfB}{\mathfrak{B}}
\newcommand{\mfC}{\mathfrak{C}}	\newcommand{\mfD}{\mathfrak{D}}
\newcommand{\mfE}{\mathfrak{E}}	\newcommand{\mfF}{\mathfrak{F}}
\newcommand{\mfG}{\mathfrak{G}}	\newcommand{\mfH}{\mathfrak{H}}
\newcommand{\mfI}{\mathfrak{I}}	\newcommand{\mfJ}{\mathfrak{J}}
\newcommand{\mfK}{\mathfrak{K}}	\newcommand{\mfL}{\mathfrak{L}}
\newcommand{\mfM}{\mathfrak{M}}	\newcommand{\mfN}{\mathfrak{N}}
\newcommand{\mfO}{\mathfrak{O}}	\newcommand{\mfP}{\mathfrak{P}}
\newcommand{\mfQ}{\mathfrak{Q}}	\newcommand{\mfR}{\mathfrak{R}}
\newcommand{\mfS}{\mathfrak{S}}	\newcommand{\mfT}{\mathfrak{T}}
\newcommand{\mfU}{\mathfrak{U}}	\newcommand{\mfV}{\mathfrak{V}}
\newcommand{\mfW}{\mathfrak{W}}	\newcommand{\mfX}{\mathfrak{X}}
\newcommand{\mfY}{\mathfrak{Y}}	\newcommand{\mfZ}{\mathfrak{Z}}
%Small Letters
\newcommand{\mfa}{\mathfrak{a}}	\newcommand{\mfb}{\mathfrak{b}}
\newcommand{\mfc}{\mathfrak{c}}	\newcommand{\mfd}{\mathfrak{d}}
\newcommand{\mfe}{\mathfrak{e}}	\newcommand{\mff}{\mathfrak{f}}
\newcommand{\mfg}{\mathfrak{g}}	\newcommand{\mfh}{\mathfrak{h}}
\newcommand{\mfi}{\mathfrak{i}}	\newcommand{\mfj}{\mathfrak{j}}
\newcommand{\mfk}{\mathfrak{k}}	\newcommand{\mfl}{\mathfrak{l}}
\newcommand{\mfm}{\mathfrak{m}}	\newcommand{\mfn}{\mathfrak{n}}
\newcommand{\mfo}{\mathfrak{o}}	\newcommand{\mfp}{\mathfrak{p}}
\newcommand{\mfq}{\mathfrak{q}}	\newcommand{\mfr}{\mathfrak{r}}
\newcommand{\mfs}{\mathfrak{s}}	\newcommand{\mft}{\mathfrak{t}}
\newcommand{\mfu}{\mathfrak{u}}	\newcommand{\mfv}{\mathfrak{v}}
\newcommand{\mfw}{\mathfrak{w}}	\newcommand{\mfx}{\mathfrak{x}}
\newcommand{\mfy}{\mathfrak{y}}	\newcommand{\mfz}{\mathfrak{z}}

\title{\Huge{Math 275} Notes}
\author{\huge{Jeremiah Vuong}\\Los Angeles Pierce College\\vuongjn5900@student.laccd.edu
}
\date{2024 Spring}

\begin{document}
\maketitle
\newpage

\pagebreak

\section{1}
\subsection{Slope (Direction) Fields}
Equations containing derivatives are differential equations. \\
A differential equation that describes some physical process is often called a mathematical model of the process.
\dfn{Slope (direction) field}{
  Given the differential equation $d y / d t=f(t, y)$. If we systematically evaluate $f$ over a rectangular grid of points in the ty-plane and draw a line element at each point $(t, y)$ of the grid with slope $f(t, y)$, then the collection of all these line elements is called a slope (direction) field of the differential equation $d y / d t=f(t, y)$
} \noindent
We can graph undefined slopes in a slope field by using a vertical line; horizontal for 0.
\qs{pg. 8: \#6}{
  Write down a differential equation of the form $\mathrm{dy} / \mathrm{dt}=\mathrm{ay}+\mathrm{b}$ whose solutions diverge from $\mathrm{y}=2$.
}
\sol \\
Given that the equation is dependent on just $y$, we can say that it is an autonomous differential equation (DE).
We can subsequently solve for the DE by setting $\frac{dy}{dt} = f(y) = 0$, such that its solution is its equilibrium solution.
Since $y=2$, $\boxed{\frac{dy}{dt} = y-2}$. Thus, any number that is less than 2 will result in the solution's phase portrait diverging (-) from $y=2$ --- vice versa for numbers greater than 2.
\subsection{tfi}
\subsection{Classification on DEs}
The order of a DE is the order of the highest derivative that appears in the equation.
\dfn{Linear ODE}{The ODE $F\left(t, y, y^{\prime}, \ldots, y^{(n)}\right)=0$ is said to be linear if $F$ is a linear function of the variables $y, y^{\prime}, \ldots, y^{(n)}$. Thus the general linear ODE of order $n$ is $a_0(t) y^{(n)}+a_1(t) y^{(n-1)}+\ldots+a_n(t) y=g(t)$.}
\qs{pg. 22}{
  Determine the order of the given DE; also state whether the equation is linear or nonlinear: \\
  1) $t^2 y^{\prime \prime}+t y^{\prime}+2 y=\sin t$ \\
  2) $\left(1+y^2\right) y^{\prime \prime}+t y^{\prime}+y=e^t$
}
\sol \\ 
1) The order of the DE is 2, and it is linear: noticed how all the coefficents of n derivatives of $y$ are functions of $t$. \\
2) The order of the DE is 2, and it is nonlinear: noticed how the coefficents of n derivatives of $y$ are functions of $t$ and $y$.

\dfn{Nth-order IDE solutions}{Any function $h$, defined on an interval and possessing at least $n$ derivatives that are continuous on this interval, which substituted into an nth-order ODE reduces the equation to an identity, is said to be a solution of the equation on the interval.}
\qs{pg. 22: \#9)}{
  Verify that the functions $y_1(t)=t^{-2}$ and $y_2(t)=t^{-2} \ln t$ are solutions of the DE
$$ t^2 y^{\prime \prime}+5 t y^{\prime}+4 y=0 $$
}
\sol \\
We can verify that $y_1(t)$ and $y_2(t)$ are solutions of the DE by substituting them into the DE and verifying that the equation holds true. \\
\begin{align*}
  y_1 &= t^{-2} \\
  y_1^{\prime} &= -2t^{-3} \\
  y_1^{\prime \prime} &= 6t^{-4} \\
\end{align*}
Substituting: $t^2 y^{\prime \prime}+5 t y^{\prime}+4 y  = t^2(6t^{-4})+5t(-2t^{-3})+4t^{-2} = 0$ \\
For homework 1, we can verify that $y_2(t)$ is a solution of the DE.

\qs{\#12}{
  Determine the values of $r$ for which $y^{\prime \prime}+y^{\prime}-6 y=0$ has solutions of the form $y=e^{r t}$
}
\sol \\
We can solve for the values of $r$ by substituting $y=e^{rt}$ into the DE and solving for $r$.
$y=e^{rt}$, $y^{\prime}=re^{rt}$, and $y^{\prime \prime}=r^2e^{rt}$. Substituting: $r^2e^{rt}+re^{rt}-6e^{rt}=0$.
Factoring out $e^{rt}$, we get $e^{rt}(r^2+r-6)=0$. Thus, $r^2+r-6=0$, and $\boxed{r=2, -3}$.

\qs{\#14} {
  Determine the values of $r$ for which $t^2 y^{\prime \prime}+4 t y^{\prime}+2 y=0$ has solutions of the form $y=t^r$ for $t>0$
}
\sol \\
$y=t^r$, $y^{\prime}=rt^{r-1}$, and $y^{\prime \prime}=r(r-1)t^{r-2}$.
Substituting: $t^2(r(r-1)t^{r-2})+4t(rt^{r-1})+2t^r=0$.
Factoring out $t^r$, we get $t^r[r(r-1)+4r+2]=0$. $t > 0$ such that $t^r \neq 0$, and $r(r-1)+4r+2=0$. Thus, $r^2+3r+2=0$, and $\boxed{r=-1, -2}$.
\section{tfi}
\subsection{Linear DEs}
\dfn{First-order linear DE}{
  An ODE of the form $$\frac{dy}{dt}+p(t) y=g(t)$$ is called a first-order linear differential equation in standard form
}
\ex{Solving a Linear First-Order DE}{
  i) Put the linear equation in standard form. \\
  ii) From the standard form of the equation identify $\mathrm{p}(\mathrm{t})$ and then find the integrating factor $e^{\int p(t) d t}$. No constant need be used in evaluating the indefinite integral in the exponent. \\
  iii) Multiply both sides of the standard form equation by the integrating factor. The left-hand side of the resulting equation is automatically the derivative of the product of the integrating factor and $\mathrm{y}: \quad \frac{d}{d t}\left[e^{\int p(t) d t} y\right]=e^{\int p(t) d t} g(t)$. \\
  iv) Integrate both sides of the last equation and solve for $y$
}
\nt{
  Integrating Factor: $\mu = e^{\int p(t) dt}$
}
\qs{pg. 31: \#6)}{
  Find the general solution of the differential equation: $t y^{\prime}-\mathrm{y}=\mathrm{t}^2 \mathrm{e}^{-\mathrm{t}}$, where $\mathrm{t}>0$
}
\sol \\
Rewrite the DE: $t \frac{dy}{dt} - y = t^2 e^{-t}$.
Divide both sides by $t$, such that $\frac{dy}{dt} - \frac{1}{t}y = t e^{-t}$.
We can then identify $p(t) = -\frac{1}{t}$, such that $e^{\int p(t) dt} = e^{\int -\frac{1}{t} dt} = e^{-\ln|t|} = e^{ln(t^{-1})} = \frac{1}{t}$.
Therefore, $\frac{d}{dt}[\frac{1}{t}y] = \frac{1}{t} \cdot t e^{-t} = e^{-t}$.
Integrating both sides, $\int \frac{d}{dt}[\frac{1}{t}y] = \int e^{-t} \implies \frac{1}{t}y = -e^{-t} + C \implies \boxed{y=-te^{-t} + Ct}$

\qs{\#11)}{
  Find the solution of the initial value problem: $y^{\prime}+\frac{2}{t} y=\frac{\cos t}{t^2}, y(\pi)=0$, where $\mathrm{t}>0$
}
\sol \\
We know that $p(t) = \frac{2}{t}$, such that $\mu = e^{\int \frac{2}{t} dt} = e^{2\ln|t|} = t^2$.
Multiplying both sides by $\mu$, we get $\frac{d}{dt}[t^2 y] = t^2 \frac{\cos t}{t^2} = \cos t$.
Integrating both sides, we get $\int \frac{d}{dt} [t^2 y] = \int \cos t \implies t^2 y = \sin t + C$.
Such that, $y=\frac{\sin t}{t^2} + \frac{C}{t^2}$. Given that $y(\pi) = 0$, we can solve for $C$.
$0 = \frac{\sin \pi}{\pi^2} + \frac{C}{\pi^2} \implies C = 0$. Therefore, $\boxed{y=\frac{\sin t}{t^2}}$.

\subsection{Separable DEs}
\dfn{Seperable DE}{
  An ODE written in the differential form $M(x) d x+N(y) d y=0$ is said to be separable since terms involving each variable may be placed on opposite sides of the equation. A separable equation can be solved by integrating the functions $M$ and $N$.
}
\qs{pg. 38: \#4)}{
  Solve the DE:$\quad x y^{\prime}=\left(1-y^2\right)^{1 / 2}$
}
\sol \\
We can rewrite the DE as $x \frac{dy}{dx} = \sqrt{1-y^2}$,
we can then seperate the variables and integrate both sides.
$ x dy = \sqrt{1-y^2} dx \implies \int dy \frac{1}{\sqrt{1-y^2}} = \int dx \frac{1}{x}$.
Such that, $\arcsin(y) = \ln|x|+C$. Thus, $\boxed{y=\sin(\ln|x|+C)}$, we call this the general solution.

\dfn{General Order n-th ODE}{
  Trying to solve the ODE $\frac{d^n y}{d x^n}=f\left(x, y, y^{\prime}, \ldots, y^{(n-1)}\right)$ subject to the conditions $\quad y\left(x_0\right)=y_0, y^{\prime}\left(x_0\right)=y_1, \ldots, y^{(n-1)}\left(x_0\right)=y_{n-1}$, where $\mathrm{y}_0, \mathrm{y}_1, \ldots, \mathrm{y}_{\mathrm{n}-1}$ are arbitrary real constants, is called an nth-order initial-value problem $(I V P)$. The values of $\mathrm{y}(\mathrm{x})$ and its first $\mathrm{n}-1$ derivatives at $\mathrm{x}_0$ are called initial conditions (IC).
}

\qs{pg. 38: \#16)}{
  Find the solution of the initial value problem $\sin (2 x) d x+\cos (3 y) d y=0, \quad y\left(\frac{\pi}{2}\right)=\frac{\pi}{3}$ in explicit form.
}
\sol \\
We can solve the DE by seperating the variables and integrating both sides.
$\sin(2x)dx = -\cos(3y)dy \implies \int \sin(2x)dx = -\int \cos(3y)dy$.
Such that, $-\frac{1}{2}\cos(2x) = -\frac{1}{3}\sin(3y) + C$.
We can solve for $C$ by using the initial condition $y(\frac{\pi}{2}) = \frac{\pi}{3}$.
$-\frac{1}{2}\cos(\pi) = -\frac{1}{3}\sin(\pi) + C \implies C = \frac{1}{2}$.
Thus, $-\frac{1}{2} \cos(2x) = \frac{1}{3}\sin(3y) + \frac{1}{2}$.
\begin{align*}
  6[-\frac{1}{2} \cos(2x)] &= 6[-\frac{1}{3}\sin(3y) + \frac{1}{2}] \\
  -3\cos(2x) &= 2\sin(3y) + 3 \\
  -3-3cos(2x) &= -2\sin(3y) \\
  (3\cos(2x)+3) \cdot \frac{1}{2} &= \sin(3y) \\
  \Aboxed{y &= \frac{1}{3} \cdot \arcsin(\frac{3+3\cos(2x)}{2})}
\end{align*}

\dfn{Homogeneous DEs}{
  A first-order DE in differential form $M(x, y) d x+N(x, y) d y=0$ is said to be homogeneous if both coefficient functions $M$ and $N$ have the same degree.
}
\noindent
Either of the substitutions $y=u x$ or $x=v y$, where $u$ and $v$ are new dependent variables, will reduce a homogeneous equation to a separable first-order differential equation.
\double
Although either of the indicated substitutions can be used for every homogeneous differential equation, in practice we try $x=v y$ whenever the function $M$ is simpler than $N$, and $y=u x$ whenever $N$ is simpler than $M$.


\qs{pg. 39: \#26)}{
  Solve: $$\frac{d y}{d x}=\frac{x^2+x y+y^2}{x^2}$$
}
\sol \\
Seperating the differential, $x^2 dy = (x^2+xy+y^2)dx \implies 0 = (x^2 + xy+ y^2) dx - x^2 dy$.
Let $y=ux$, such that $dy = udx + xdu$. Where $0 = (x^2 + x(ux) + (ux)^2)dx - x^2(udx + xdu)$.
Distributing,
\begin{align*}
0 &= x^2 dx + ux^2 dx + u^2x^2 dx - ux^2dx - x^3 du \\
x^3 du &= x^2 dx + u^2x^2 dx \\
\int \frac{du}{u^2+1} &= \int \frac{dx}{x} \\
\arctan(u) &= \ln|x| + C \\
\arctan(\frac{y}{x}) &= \ln|x| + C \\
\Aboxed{y &= x\tan(\ln|x|+C)}
\end{align*}

\qs{\#28) }{
  Solve: $$\frac{d y}{d x}=\frac{4 y-3 x}{2 x-y}$$
}
\sol \\
Cross multplying, $(2x-y) dy = (4y-3x)dx \implies 0 = (4y-3x)dx + (y-2x)dy$.
Let $x=vy$, such that $dx = vdy + ydv$. Where $0 = (4y-3vy)(ydy+ydv) + (y-2vy)dy$.
Distributing,
\begin{align*}
  0 &= 4vydy+4y^2dv-3v^2ydy-3vy^2dv+ydy-2vydy \\
  0 &= 2vydy + 4y^2dv-3v^2ydy-3vy^2dv+ydy \\
  3vy^2dv-4y^2dv &= 2vydy-3vy^2dy-ydy \\
  y^2(3v-4)dv &= y(2v-3v^2+1)dy \\ 
  y^2(3v-4)dv &= -y(3v^2-2v-1) dy \\
  \int \frac{3v-4}{(v-1)(3v+1)} dv &= -\int \frac{1}{y} dy \\
\end{align*}
Breaking up the L.H.S. into partial fractions,
\begin{align*}
  \frac{3v-4}{(v-1)(3v+1)} &= \frac{A}{v-1} + \frac{B}{3v+1} \\
  3v-4 &= A(3v+1) + B(v-1) \\
\end{align*}
Let $v = -\frac{1}{3}$, such that $3(-\frac{1}{3})-4=A(0)+B(-\frac{1}{3}-1) \implies B = \frac{15}{4}$. \\
Let $v = 1$, such that $3(1)-4=A(3+1)+B(0) \implies A = -\frac{1}{4}$. \\
\begin{align*}
  \int \frac{-\frac{1}{4}}{v-1} + \frac{\frac{15}{4}}{3v+1} dv &= -\int \frac{1}{y} dy \\
  -\frac{1}{4} \ln|v-1| + \frac{15}{12} \ln|3v+1| &= -\ln|y| + C \\
  -\frac{1}{3} \ln|\frac{x}{y} - 1| + \frac{5}{4} \ln|\frac{3x}{y}+1| &= -\ln|y| + C \\
  \Aboxed{-\ln|\frac{x-y}{y}| + 5 \ln|\frac{3x+y}{y}| &= -4\ln|y| + C} \\
\end{align*}

\subsection{Modeling with First Order DEs}

\dfn{Initial Value Problem (IVP)}{
  The IVP $\quad \frac{dx}{dt} = kx, \quad x\left(t_0\right)=x_0 \quad$ where $k$ is a constant of proportionality, serves as a model for diverse phenomena involving either growth or decay. The constant $k$ is called the growth or decay constant.
}

\begin{align*}
  \frac{dx}{dt} &= kx \\
  \frac{1}{x} dx &= k dt \\
  \int \frac{1}{x} dx &= \int k dt \\
  \ln|x| &= kt + C \\
  |x| &= e^{kt}e^C = Ce^{kt} \\
  x &= \pm Ce^{kt} = Ce^{kt} \\
  x_0 &= Ce^{k(0)} = C \\
  \Aboxed{x &= x_0 e^{kt}}
\end{align*}

\qs{IVP/Exponential}{
  A culture initially has $P_0$ number of bacteria. At $t=1$ hour the number of bacteria is measured to be $1.5 \mathrm{P}_0$. If the rate of growth is proportional to the number of bacteria $P(t)$ present at time $t$, determine the time necessary for the number of bacteria to triple.
}
\sol \\
$\frac{dP}{dt} = kP$, where $P(0) = P_0$, $P(1) = 1.5P_0$. We want to solve for $t$ such that $P(t) = 3P_0$.
We know that $P(t) = P_0 e^{kt}$, such that $1.5P_0 = P_0 e^{k(1)} \implies 1.5 = e^k \implies ln(1.5) = k \approx 0.4055$.
Such that, $P = P_0 e^{0.4055t} \implies 3P_0 = P_0 e^{0.4055t} \implies 3 = e^{0.4055t} \implies ln(3) = 0.4055t \implies t = \frac{\ln 3}{0.4055} = \boxed{2.7 \text{hours.}}$

\dfn{Linear First-Order DEs}{
  The mixing of two fluids sometimes gives rise to a linear first-order differential equation. When two brine solutions are mixed, if we assume that the rate $A^{\prime}(t)$ at which the amount of salt in the mixing tank changes is a net rate, then
$$
\begin{aligned}
& \mathrm{dA} / \mathrm{dt}=\text { (input rate of salt) }- \text { (output rate of salt) } \\
& =\quad R_{\text {in }} \quad-\quad R_{\text {out }} \\
&
\end{aligned}
$$
}

\qs{Linear FODE}{
  Suppose that a large mixing tank initially holds 300 gallons of brine. Another brine solution is pumped into the large tank at a rate of 3 gallons/minute, and the concentration of the salt in this inflow is 2 pounds/gallon. When the solution in the tank is well stirred, it is pumped out at the same rate as the entering solution. If 50 pounds of salt were dissolved initially in the 300 gallons, how much salt is in the tank after a long time?
  \imgg{qs2.8}{0.4}
}
\sol \\
We can model the ROC of the salt in the tank as $\frac{dA}{dt} = R_{in} - R_{out}$,
where $\frac{dA}{dt}$ is modeled as lbs/min (such that both in and out have to be the same units).
Such that, $R_{in} = (3 \frac{gal}{min})(2 \frac{lbs}{gal}) = 6 \frac{lbs}{min}$.
We can then model $R_{out} = (3 \frac{gal}{min})(\frac{A}{300} \frac{lbs}{gal}) = \frac{A}{100} \frac{lbs}{min}$.
Thus, $\frac{dA}{dt} = 6 - \frac{A}{100}$, where $A(0) = 50$.
We can rewrite the DE as $\frac{dA}{dt} + \frac{1}{100}A = 6$, such that $p(t) = \frac{1}{100}$ and $g(t) = 6$.
Where $\mu = e^{\int \frac{1}{100} dt} = e^{\frac{1}{100}t}$.
Such that, $\frac{d}{dt} [e^{\frac{1}{100}t}A] = 6e^{\frac{1}{100}t} \implies \int \frac{d}{dt} [e^{\frac{1}{100}t}A] = \int 6e^{\frac{1}{100}t}
\implies e^{\frac{1}{100}t} A = 6 \cdot 100 e^{\frac{1}{100}t} + C \implies A = 600 + \frac{C}{e^{\frac{1}{100}t}}$.
Given that $A(0) = 50$, we can solve for $C$ such that $50 = 600 + \frac{C}{e^0} \implies C = -550$.
Therefore, $A=600-\frac{550}{e^{\frac{1}{100}t}}$, where after a long time, $\lim_{t \to \infty} A = \boxed{600 \text{ lbs}}$.

\end{document}