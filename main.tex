\documentclass{article}

\input{preamble}
\input{macros}
\input{letterfonts}

\title{\Huge{Math 275} Notes}
\author{\huge{Jeremiah Vuong}\\Los Angeles Pierce College\\vuongjn5900@student.laccd.edu
}
\date{2024 Spring}

\begin{document}
\maketitle
\newpage

\pagebreak

\section{1}
\subsection{Slope (Direction) Fields}
Equations containing derivatives are differential equations. \\
A differential equation that describes some physical process is often called a mathematical model of the process.
\dfn{Slope (direction) field}{
  Given the differential equation $d y / d t=f(t, y)$. If we systematically evaluate $f$ over a rectangular grid of points in the ty-plane and draw a line element at each point $(t, y)$ of the grid with slope $f(t, y)$, then the collection of all these line elements is called a slope (direction) field of the differential equation $d y / d t=f(t, y)$
} \noindent
We can graph undefined slopes in a slope field by using a vertical line; horizontal for 0.
\qs{pg. 8: \#6}{
  Write down a differential equation of the form $\mathrm{dy} / \mathrm{dt}=\mathrm{ay}+\mathrm{b}$ whose solutions diverge from $\mathrm{y}=2$.
}
\sol \\
Given that the equation is dependent on just $y$, we can say that it is an autonomous differential equation (DE).
We can subsequently solve for the DE by setting $\frac{dy}{dt} = f(y) = 0$, such that its solution is its equilibrium solution.
Since $y=2$, $\boxed{\frac{dy}{dt} = y-2}$. Thus, any number that is less than 2 will result in the solution's phase portrait diverging (-) from $y=2$ --- vice versa for numbers greater than 2.
\subsection{tfi}
\subsection{Classification on DEs}
The order of a DE is the order of the highest derivative that appears in the equation.
\dfn{Linear ODE}{The ODE $F\left(t, y, y^{\prime}, \ldots, y^{(n)}\right)=0$ is said to be linear if $F$ is a linear function of the variables $y, y^{\prime}, \ldots, y^{(n)}$. Thus the general linear ODE of order $n$ is $a_0(t) y^{(n)}+a_1(t) y^{(n-1)}+\ldots+a_n(t) y=g(t)$.}
\qs{pg. 22}{
  Determine the order of the given DE; also state whether the equation is linear or nonlinear: \\
  1) $t^2 y^{\prime \prime}+t y^{\prime}+2 y=\sin t$ \\
  2) $\left(1+y^2\right) y^{\prime \prime}+t y^{\prime}+y=e^t$
}
\sol \\ 
1) The order of the DE is 2, and it is linear: noticed how all the coefficents of n derivatives of $y$ are functions of $t$. \\
2) The order of the DE is 2, and it is nonlinear: noticed how the coefficents of n derivatives of $y$ are functions of $t$ and $y$.

\dfn{Nth-order IDE solutions}{Any function $h$, defined on an interval and possessing at least $n$ derivatives that are continuous on this interval, which substituted into an nth-order ODE reduces the equation to an identity, is said to be a solution of the equation on the interval.}
\qs{pg. 22: \#9)}{
  Verify that the functions $y_1(t)=t^{-2}$ and $y_2(t)=t^{-2} \ln t$ are solutions of the DE
$$ t^2 y^{\prime \prime}+5 t y^{\prime}+4 y=0 $$
}
\sol \\
We can verify that $y_1(t)$ and $y_2(t)$ are solutions of the DE by substituting them into the DE and verifying that the equation holds true. \\
\begin{align*}
  y_1 &= t^{-2} \\
  y_1^{\prime} &= -2t^{-3} \\
  y_1^{\prime \prime} &= 6t^{-4} \\
\end{align*}
Substituting: $t^2 y^{\prime \prime}+5 t y^{\prime}+4 y  = t^2(6t^{-4})+5t(-2t^{-3})+4t^{-2} = 0$ \\
For homework 1, we can verify that $y_2(t)$ is a solution of the DE.

\qs{\#12}{
  Determine the values of $r$ for which $y^{\prime \prime}+y^{\prime}-6 y=0$ has solutions of the form $y=e^{r t}$
}
\sol \\
We can solve for the values of $r$ by substituting $y=e^{rt}$ into the DE and solving for $r$.
$y=e^{rt}$, $y^{\prime}=re^{rt}$, and $y^{\prime \prime}=r^2e^{rt}$. Substituting: $r^2e^{rt}+re^{rt}-6e^{rt}=0$.
Factoring out $e^{rt}$, we get $e^{rt}(r^2+r-6)=0$. Thus, $r^2+r-6=0$, and $\boxed{r=2, -3}$.

\qs{\#14} {
  Determine the values of $r$ for which $t^2 y^{\prime \prime}+4 t y^{\prime}+2 y=0$ has solutions of the form $y=t^r$ for $t>0$
}
\sol \\
$y=t^r$, $y^{\prime}=rt^{r-1}$, and $y^{\prime \prime}=r(r-1)t^{r-2}$.
Substituting: $t^2(r(r-1)t^{r-2})+4t(rt^{r-1})+2t^r=0$.
Factoring out $t^r$, we get $t^r[r(r-1)+4r+2]=0$. $t > 0$ such that $t^r \neq 0$, and $r(r-1)+4r+2=0$. Thus, $r^2+3r+2=0$, and $\boxed{r=-1, -2}$.
\section{tfi}
\subsection{Linear DEs}
\dfn{First-order linear DE}{
  An ODE of the form $$\frac{dy}{dt}+P(t) y= Q(t)$$ is called a first-order linear differential equation in standard form.
  $$ I(x) = e^{\int P(t) dt} $$
  such that,
  $$ y = \frac{1}{I(t)} \int I(t) Q(t) dt + C $$
}
\qs{pg. 31: \#6)}{
  Find the general solution of the differential equation: $t y^{\prime}-\mathrm{y}=\mathrm{t}^2 \mathrm{e}^{-\mathrm{t}}$, where $\mathrm{t}>0$
}
\sol \\
Rewrite the DE: $t \frac{dy}{dt} - y = t^2 e^{-t}$.
Divide both sides by $t$, such that $\frac{dy}{dt} - \frac{1}{t}y = t e^{-t}$.
We can then identify $p(t) = -\frac{1}{t}$, such that $e^{\int p(t) dt} = e^{\int -\frac{1}{t} dt} = e^{-\ln|t|} = e^{ln(t^{-1})} = \frac{1}{t}$.
Therefore, $\frac{d}{dt}[\frac{1}{t}y] = \frac{1}{t} \cdot t e^{-t} = e^{-t}$.
Integrating both sides, $\int \frac{d}{dt}[\frac{1}{t}y] = \int e^{-t} \implies \frac{1}{t}y = -e^{-t} + C \implies \boxed{y=-te^{-t} + Ct}$

\qs{\#11)}{
  Find the solution of the initial value problem: $y^{\prime}+\frac{2}{t} y=\frac{\cos t}{t^2}, y(\pi)=0$, where $\mathrm{t}>0$
}
\sol \\
We know that $p(t) = \frac{2}{t}$, such that $\mu = e^{\int \frac{2}{t} dt} = e^{2\ln|t|} = t^2$.
Multiplying both sides by $\mu$, we get $\frac{d}{dt}[t^2 y] = t^2 \frac{\cos t}{t^2} = \cos t$.
Integrating both sides, we get $\int \frac{d}{dt} [t^2 y] = \int \cos t \implies t^2 y = \sin t + C$.
Such that, $y=\frac{\sin t}{t^2} + \frac{C}{t^2}$. Given that $y(\pi) = 0$, we can solve for $C$.
$0 = \frac{\sin \pi}{\pi^2} + \frac{C}{\pi^2} \implies C = 0$. Therefore, $\boxed{y=\frac{\sin t}{t^2}}$.

\subsection{Separable DEs}
\dfn{Seperable DE}{
  An ODE written in the differential form $M(x) d x+N(y) d y=0$ is said to be separable since terms involving each variable may be placed on opposite sides of the equation. A separable equation can be solved by integrating the functions $M$ and $N$.
}
\qs{pg. 38: \#4)}{
  Solve the DE:$\quad x y^{\prime}=\left(1-y^2\right)^{1 / 2}$
}
\sol \\
We can rewrite the DE as $x \frac{dy}{dx} = \sqrt{1-y^2}$,
we can then seperate the variables and integrate both sides.
$ x dy = \sqrt{1-y^2} dx \implies \int dy \frac{1}{\sqrt{1-y^2}} = \int dx \frac{1}{x}$.
Such that, $\arcsin(y) = \ln|x|+C$. Thus, $\boxed{y=\sin(\ln|x|+C)}$, we call this the general solution.

\dfn{General Order n-th ODE}{
  Trying to solve the ODE $\frac{d^n y}{d x^n}=f\left(x, y, y^{\prime}, \ldots, y^{(n-1)}\right)$ subject to the conditions $\quad y\left(x_0\right)=y_0, y^{\prime}\left(x_0\right)=y_1, \ldots, y^{(n-1)}\left(x_0\right)=y_{n-1}$, where $\mathrm{y}_0, \mathrm{y}_1, \ldots, \mathrm{y}_{\mathrm{n}-1}$ are arbitrary real constants, is called an nth-order initial-value problem $(I V P)$. The values of $\mathrm{y}(\mathrm{x})$ and its first $\mathrm{n}-1$ derivatives at $\mathrm{x}_0$ are called initial conditions (IC).
}

\qs{pg. 38: \#16)}{
  Find the solution of the initial value problem $\sin (2 x) d x+\cos (3 y) d y=0, \quad y\left(\frac{\pi}{2}\right)=\frac{\pi}{3}$ in explicit form.
}
\sol \\
We can solve the DE by seperating the variables and integrating both sides.
$\sin(2x)dx = -\cos(3y)dy \implies \int \sin(2x)dx = -\int \cos(3y)dy$.
Such that, $-\frac{1}{2}\cos(2x) = -\frac{1}{3}\sin(3y) + C$.
We can solve for $C$ by using the initial condition $y(\frac{\pi}{2}) = \frac{\pi}{3}$.
$-\frac{1}{2}\cos(\pi) = -\frac{1}{3}\sin(\pi) + C \implies C = \frac{1}{2}$.
Thus, $-\frac{1}{2} \cos(2x) = \frac{1}{3}\sin(3y) + \frac{1}{2}$.
\begin{align*}
  6[-\frac{1}{2} \cos(2x)] &= 6[-\frac{1}{3}\sin(3y) + \frac{1}{2}] \\
  -3\cos(2x) &= 2\sin(3y) + 3 \\
  -3-3cos(2x) &= -2\sin(3y) \\
  (3\cos(2x)+3) \cdot \frac{1}{2} &= \sin(3y) \\
  \Aboxed{y &= \frac{1}{3} \cdot \arcsin(\frac{3+3\cos(2x)}{2})}
\end{align*}

\dfn{Homogeneous DEs}{
  A first-order DE in differential form $M(x, y) d x+N(x, y) d y=0$ is said to be homogeneous if both coefficient functions $M$ and $N$ have the same degree.
}
\noindent
Either of the substitutions $y=u x$ or $x=v y$, where $u$ and $v$ are new dependent variables, will reduce a homogeneous equation to a separable first-order differential equation.
\double
Although either of the indicated substitutions can be used for every homogeneous differential equation, in practice we try $x=v y$ whenever the function $M$ is simpler than $N$, and $y=u x$ whenever $N$ is simpler than $M$.


\qs{pg. 39: \#26)}{
  Solve: $$\frac{d y}{d x}=\frac{x^2+x y+y^2}{x^2}$$
}
\sol \\
Seperating the differential, $x^2 dy = (x^2+xy+y^2)dx \implies 0 = (x^2 + xy+ y^2) dx - x^2 dy$.
Let $y=ux$, such that $dy = udx + xdu$. Where $0 = (x^2 + x(ux) + (ux)^2)dx - x^2(udx + xdu)$.
Distributing,
\begin{align*}
0 &= x^2 dx + ux^2 dx + u^2x^2 dx - ux^2dx - x^3 du \\
x^3 du &= x^2 dx + u^2x^2 dx \\
\int \frac{du}{u^2+1} &= \int \frac{dx}{x} \\
\arctan(u) &= \ln|x| + C \\
\arctan(\frac{y}{x}) &= \ln|x| + C \\
\Aboxed{y &= x\tan(\ln|x|+C)}
\end{align*}

\qs{\#28) }{
  Solve: $$\frac{d y}{d x}=\frac{4 y-3 x}{2 x-y}$$
}
\sol \\
Cross multplying, $(2x-y) dy = (4y-3x)dx \implies 0 = (4y-3x)dx + (y-2x)dy$.
Let $x=vy$, such that $dx = vdy + ydv$. Where $0 = (4y-3vy)(ydy+ydv) + (y-2vy)dy$.
Distributing,
\begin{align*}
  0 &= 4vydy+4y^2dv-3v^2ydy-3vy^2dv+ydy-2vydy \\
  0 &= 2vydy + 4y^2dv-3v^2ydy-3vy^2dv+ydy \\
  3vy^2dv-4y^2dv &= 2vydy-3vy^2dy-ydy \\
  y^2(3v-4)dv &= y(2v-3v^2+1)dy \\ 
  y^2(3v-4)dv &= -y(3v^2-2v-1) dy \\
  \int \frac{3v-4}{(v-1)(3v+1)} dv &= -\int \frac{1}{y} dy \\
\end{align*}
Breaking up the L.H.S. into partial fractions,
\begin{align*}
  \frac{3v-4}{(v-1)(3v+1)} &= \frac{A}{v-1} + \frac{B}{3v+1} \\
  3v-4 &= A(3v+1) + B(v-1) \\
\end{align*}
Let $v = -\frac{1}{3}$, such that $3(-\frac{1}{3})-4=A(0)+B(-\frac{1}{3}-1) \implies B = \frac{15}{4}$. \\
Let $v = 1$, such that $3(1)-4=A(3+1)+B(0) \implies A = -\frac{1}{4}$. \\
\begin{align*}
  \int \frac{-\frac{1}{4}}{v-1} + \frac{\frac{15}{4}}{3v+1} dv &= -\int \frac{1}{y} dy \\
  -\frac{1}{4} \ln|v-1| + \frac{15}{12} \ln|3v+1| &= -\ln|y| + C \\
  -\frac{1}{3} \ln|\frac{x}{y} - 1| + \frac{5}{4} \ln|\frac{3x}{y}+1| &= -\ln|y| + C \\
  \Aboxed{-\ln|\frac{x-y}{y}| + 5 \ln|\frac{3x+y}{y}| &= -4\ln|y| + C} \\
\end{align*}

\subsection{Modeling with First Order DEs}

\dfn{Initial Value Problem (IVP)}{
  The IVP $\quad \frac{dx}{dt} = kx, \quad x\left(t_0\right)=x_0 \quad$ where $k$ is a constant of proportionality, serves as a model for diverse phenomena involving either growth or decay. The constant $k$ is called the growth or decay constant.
}

\begin{align*}
  \frac{dx}{dt} &= kx \\
  \frac{1}{x} dx &= k dt \\
  \int \frac{1}{x} dx &= \int k dt \\
  \ln|x| &= kt + C \\
  |x| &= e^{kt}e^C = Ce^{kt} \\
  x &= \pm Ce^{kt} = Ce^{kt} \\
  x_0 &= Ce^{k(0)} = C \\
  \Aboxed{x &= x_0 e^{kt}}
\end{align*}

\qs{IVP/Exponential}{
  A culture initially has $P_0$ number of bacteria. At $t=1$ hour the number of bacteria is measured to be $1.5 \mathrm{P}_0$. If the rate of growth is proportional to the number of bacteria $P(t)$ present at time $t$, determine the time necessary for the number of bacteria to triple.
}
\sol \\
$\frac{dP}{dt} = kP$, where $P(0) = P_0$, $P(1) = 1.5P_0$. We want to solve for $t$ such that $P(t) = 3P_0$.
We know that $P(t) = P_0 e^{kt}$, such that $1.5P_0 = P_0 e^{k(1)} \implies 1.5 = e^k \implies ln(1.5) = k \approx 0.4055$.
Such that, $P = P_0 e^{0.4055t} \implies 3P_0 = P_0 e^{0.4055t} \implies 3 = e^{0.4055t} \implies ln(3) = 0.4055t \implies t = \frac{\ln 3}{0.4055} = \boxed{2.7 \text{hours.}}$

\dfn{Linear First-Order DEs}{
  The mixing of two fluids sometimes gives rise to a linear first-order differential equation. When two brine solutions are mixed, if we assume that the rate $A^{\prime}(t)$ at which the amount of salt in the mixing tank changes is a net rate, then
$$
\begin{aligned}
& \mathrm{dA} / \mathrm{dt}=\text { (input rate of salt) }- \text { (output rate of salt) } \\
& =\quad R_{\text {in }} \quad-\quad R_{\text {out }} \\
&
\end{aligned}
$$
}

\qs{Linear FODE}{
  Suppose that a large mixing tank initially holds 300 gallons of brine. Another brine solution is pumped into the large tank at a rate of 3 gallons/minute, and the concentration of the salt in this inflow is 2 pounds/gallon. When the solution in the tank is well stirred, it is pumped out at the same rate as the entering solution. If 50 pounds of salt were dissolved initially in the 300 gallons, how much salt is in the tank after a long time?
  \imgg{qs2.8}{0.4}
}
\sol \\
We can model the ROC of the salt in the tank as $\frac{dA}{dt} = R_{in} - R_{out}$,
where $\frac{dA}{dt}$ is modeled as lbs/min (such that both in and out have to be the same units).
Such that, $R_{in} = (3 \frac{gal}{min})(2 \frac{lbs}{gal}) = 6 \frac{lbs}{min}$.
We can then model $R_{out} = (3 \frac{gal}{min})(\frac{A}{300} \frac{lbs}{gal}) = \frac{A}{100} \frac{lbs}{min}$.
Thus, $\frac{dA}{dt} = 6 - \frac{A}{100}$, where $A(0) = 50$.
We can rewrite the DE as $\frac{dA}{dt} + \frac{1}{100}A = 6$, such that $p(t) = \frac{1}{100}$ and $g(t) = 6$.
Where $\mu = e^{\int \frac{1}{100} dt} = e^{\frac{1}{100}t}$.
Such that, $\frac{d}{dt} [e^{\frac{1}{100}t}A] = 6e^{\frac{1}{100}t} \implies \int \frac{d}{dt} [e^{\frac{1}{100}t}A] = \int 6e^{\frac{1}{100}t}
\implies e^{\frac{1}{100}t} A = 6 \cdot 100 e^{\frac{1}{100}t} + C \implies A = 600 + \frac{C}{e^{\frac{1}{100}t}}$.
Given that $A(0) = 50$, we can solve for $C$ such that $50 = 600 + \frac{C}{e^0} \implies C = -550$.
Therefore, $A=600-\frac{550}{e^{\frac{1}{100}t}}$, where after a long time, $\lim_{t \to \infty} A = \boxed{600 \text{ lbs}}$.

\dfn{Newton's Law of Cooling}{
  $$ \frac{d T}{d t}=k\left(T-T_m\right) $$
  where $k$ is the constant of proportionality, $\mathrm{T}(\mathrm{t})$ is the temperature of the object for $t>0$, and $T_m$ is the temperature of the medium surrounding the object.
}

\qs{Newton's Law of cooling}{
  When a cake is removed from an oven, its temperature is measured at 300 degrees F.
  Three minutes later its temperature is 200 degrees F. How long will it take to cool to a temperature of 75 degrees $\mathrm{F}$,
  if the room temperature is 70 degrees $\mathrm{F}$?
}
\sol \\
We can set up a DE, $\frac{dT}{dt} = k(T-70)$, where $T(0) = 300$ and $T(3) = 200$.
Seperating the variables, $\frac{dT}{T-70} = k dt \implies \int \frac{dT}{T-75} = \int k dt
\implies \ln|T-70| = kt + C$.
Where at $T(0) = 300 \implies \ln 230 = C$.
Also at $T(3) = 200 \implies \ln 130 = 3k + \ln 230 \implies k = \frac{\ln 130 - \ln 230}{3} \approx -0.19018$.
Raising both by $e$,
\begin{align*}
  e^{ln|T-70|} &= e^{-0.19018t + \ln 230} \\
  T-70 &= 230 e^{-0.19018t} \\
  T &= 70 + 230 e^{-0.19018t} \\
  75 &= 70 + 230 e^{-0.19018t} \\
  5 &= 230 e^{-0.19018t} \\
  \frac{5}{230} &= e^{-0.19018t} \\
  ln (\frac{1}{46}) &= ln(e^{-0.19018t}) \\
  t &= \frac{\ln(\frac{1}{46})}{-0.19018} \approx \boxed{20.1 \text{ minutes}}
\end{align*}

\subsection{Exact DEs}

\dfn{Exact Differential}{
  A differential expression $M(x, y) d x+N(x, y) d y=0$ is an exact differential if it corresponds to the differential of some function $f(x, y)$. A first-order differential equation of the form
  $$ M(x, y) d x+N(x, y) d y=0 $$
  is said to be an exact equation if the expression on the left-hand side is an exact differential.
  \double
  A necessary and sufficient condition that $M(x, y) d x+N(x, y) d y$ be an exact differential is $\frac{\partial M}{\partial y}=\frac{\partial N}{\partial x}$.
}

\qs{pg. 75: \#6)}{Determine whether the equation
$$ \left(y e^{x y} \cos (2 x)-2 e^{x y} \sin (2 x)+2 x\right)+\left(x e^{x y} \cos (2 x)-3\right) y^{\prime}=0 $$
is exact. If it is exact, find the solution.
}
\sol \\
$(y e^{xy} \cos (2x) - 2 e^{xy} \sin (2x) +2x) dx + (x e^{xy} \cos (2x) - 3) dy = 0$, where $M$ is the first term and $N$ is the second term.
\begin{align*}
  \del{M}{y} &= e^{xy} \cos(2x) + yx e^{xy} \cos(2x) - 2xe^{xy} \sin(2x) \\
  \del{N}{x} &= e^{xy} \cos(2x) + yx e^{xy} \cos(2x) - 2xe^{xy} \sin(2x)
\end{align*}
Therefore they are exact such that $M = \del{f}{x}$ and $N = \del{f}{y}$.
\begin{align*}
  \int \del{f}{y} &= \int (x e^{xy} \cos (2x) -3) dy \\
  f(x,y) &= x \cos (2x) \frac{e^{xy}}{x} -3y + g(x) \\
  \del{f}{x} &= ye^{xy} \cos(2x) - 2e^{xy} \sin(2x) + g'(x) \\
  &= ye^{xy} \cos(2x) - 2e^{xy} \sin(2x) + 2x \implies g'(x) = 2x \implies g(x) = x^2 \\
  f(x,y) &= \boxed{e^{xy} \cos(2x)  -3y + x^2 = C}
\end{align*}

\nt{
  $$\frac{d}{dx} (fgh) = f'gh + fg'h + fgh'$$
}

\qs{\#10)}{Solve the initial value problem:
$$\left(9 x^2+y-1\right)-(4 y-x) y^{\prime}=0, \quad y(1)=0 $$
}

\sol \\
$(9x^2+y-1)dx + (x-4y)dy = 0$, $y(1) = 0$, where $M$ is the first term and $N$ is the second term. \\
$\del{M}{y} = 1 = \del{N}{x} \implies$ exact.
\begin{align*}
  \int \del{f}{x} &= \int (9x^2+y-1) dx \\
  f(x,y) &= 3x^3 + xy - x + g(y) \\
  \del{f}{y} &= x + g'(y) = x - 4y \\
  \implies g(y) &= -2y^2 \\
  f(x,y) = C &= 3x^3 + xy - x - 2y^2 \\
  y(1)= 0 \implies 2 &= C \\
  \Aboxed{3x^3 + xy -x -2y^2 &= 2}
\end{align*}

\qs{\#12) }{Find the value of $b$ for which the equation
$$ \left(y e^{2 x y}+x\right)+b^{2 x y} y^{\prime}=0 $$
is exact, and then solve it using that value of $b$.}
\sol \\
$(y e^{2xy} + x)dx + b^{2xy}dy = 0$, where $M$ is the first term and $N$ is the second term. \\
\begin{align*}
  \del{M}{y} &= e^{2xy} + 2xy e^{2xy} \\
  \del{N}{x} &= b [e^{2xy} + 2xy e^{2xy}] \implies b = 1 \\
\end{align*}
Such that the equation is exact when $b=1$.
\begin{align*}
  \int \del{f}{y} &= \int (xe^{2xy}) dy \\
  f(x,y) &= \frac{1}{2}e^{2xy} + g(x) \\
  \del{f}{x} &= \frac{1}{2}(2y e^{2xy}) + g'(x) \\ 
  \del{f}{x} &= y e^{2xy} + g'(x) = y e^{2xy} + x \\
  &\implies g'(x) = x \implies g(x) = \frac{1}{2}x^2 \\
  f(x,y) &= C = \frac{1}{2}e^{2xy} + \frac{1}{2}x^2 \\
  \Aboxed{C &= e^{2xy} + x^2}
\end{align*}

\subsection{Euler's Method}
\dfn{Euler's Method}{
  For small $\Delta t, \frac{d y}{d t} \approx \frac{\Delta y}{\Delta t}$, or equivalently, $\Delta y \approx \frac{d y}{d t} \Delta t$.
}

\qs{pg. 82: \#4}{
  Given $y^{\prime}=3 \cos t-2 y, \quad y(0)=0$. \\ 
a) Find approximate values of the solution of the IVP at $\mathrm{t}=0.1,0.2,0.3$, and 0.4 using Euler's method with $h=\Delta t=0.1$. \\
d) Find the actual solution of the IVP and evaluate the solution at $t=0.1,0.2,0.3$, and 0.4 . Compare these values with the results in part a).
}
\sol a) We have that $\Delta y = \frac{dy}{dt} \Delta t$ such that, $\Delta y = (3 \cos t - 2y) (0.1)$.

\begin{center}
\begin{tabular}{l|l|l}
  $t$ & $y$    & $\Delta y$ \\
  0   & 0      & 0.3        \\
  0.1 & 0.3    & 0.2385     \\
  0.2 & 0.5385 & 0.1863     \\
  0.3 & 0.7248 & 0.1416     \\
  0.4 & 0.8664  &           
\end{tabular}
\end{center}
\noindent
d) Given $\frac{dy}{dt} + 2y  = 3\cos t$, $y(0) =0$.
Such that $I(t) = e^{\int 2 dt} = e^{2t}$.
\begin{align*}
  \frac{d}{dt} [e^{2t}y] &= 3 e^{2t} \cos t \\
  y &= \frac{1}{e^{2t}} \int e^{2t} 3 \cos t dt \\
\end{align*}
Let $u = \cos t \implies du = -\sin t dt$, $dv = 3 e^{2t} \implies v = \frac{3}{2} e^{2t}$
$$ \int e^{2t} 3 \cos t dt = \frac{3}{2} e^{2t} \cos t + \frac{3}{2} \int e^{2t} \sin t dt $$
Let $u = \sin t \implies du = \cos t dt$, $dv = e^{2t} \implies v = \frac{1}{2} e^{2t}$.
\begin{align*}
&= \frac{3}{2} e^{2t} \cos t + \frac{3}{2} \left( \frac{1}{2} e^{2t} \sin t - \frac{1}{2} \int e^{2t} \cos t dt \right) \\
3 \int e^{2t} \cos t dt &= \frac{3}{2} e^{2t} \cos t + \frac{3}{4} e^{2t} \sin t - \frac{3}{4} \int e^{2t} \cos t dt \\
\frac{15}{4} \int e^{2t} \cos t dt &= \frac{3}{2} e^{2t} \cos t + \frac{3}{4} e^{2t} \sin t \\
\int e^{2t} \cos t dt &= \frac{2}{5} e^{2t} \cos t + \frac{1}{5} e^{2t} \sin t \\
y &= \frac{3}{e^{2t}} [\frac{2}{5} e^{2t} \cos t + \frac{1}{5} e^{2t} \sin t] + C \\
y &= \frac{6}{5} \cos t + \frac{3}{5} \sin t + Ce^{-2t} \\
0 &= \frac{6}{5} \cos 0 + \frac{3}{5} \sin 0 + Ce^{-2(0)} \implies C = -\frac{6}{5} \\
\Aboxed{y &= \frac{6}{5} \cos t + \frac{3}{5} \sin t - \frac{6}{5} e^{-2t}}
\end{align*}
Testing $y(0.4) = \frac{6}{5} \cos 0.4 + \frac{3}{5} \sin 0.4 - \frac{6}{5} e^{-2(0.4)} \boxed{\approx 0.7997}$. \\

\qs{\#10}{
  Use Euler's method to find approximate values of the solution of the IVP $\quad \mathrm{y}^{\prime}=\mathrm{y}(3-\mathrm{ty}), \quad \mathrm{y}(0)=0.5, \quad$ at $\mathrm{t}=0.5$ with $h=\Delta t=0.1$.
}
\sol We have that $\Delta y = y(3-ty) (0.1)$
\begin{center}
  \begin{tabular}{l|l|l}
    $t$ & $y$    & $\Delta y$ \\
    0   & 0.5    & 0.15       \\
    0.1 & 0.65   & 0.1908     \\
    0.2 & 0.8408 & 0.2381     \\
    0.3 & 1.0789 & 0.2887     \\
    0.4 & 1.3676 & 0.335      \\
    0.5 & 1.7031 & 
  \end{tabular}
  \end{center}
$\boxed{y(0.5) \approx 1.7031}$.
\end{document}