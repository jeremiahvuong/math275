\documentclass{article}

\input{preamble}
\input{macros}
\input{letterfonts}

\title{\Huge{Math 275} Notes}
\author{\huge{Jeremiah Vuong}\\Los Angeles Pierce College\\vuongjn5900@student.laccd.edu
}
\date{2024 Spring}

\begin{document}
\maketitle
\newpage

\pagebreak

\section{1}
\subsection{Slope (Direction) Fields}
Equations containing derivatives are differential equations. \\
A differential equation that describes some physical process is often called a mathematical model of the process.
\dfn{Slope (direction) field}{
  Given the differential equation $d y / d t=f(t, y)$. If we systematically evaluate $f$ over a rectangular grid of points in the ty-plane and draw a line element at each point $(t, y)$ of the grid with slope $f(t, y)$, then the collection of all these line elements is called a slope (direction) field of the differential equation $d y / d t=f(t, y)$
} \noindent
We can graph undefined slopes in a slope field by using a vertical line; horizontal for 0.
\qs{pg. 8: \#6}{
  Write down a differential equation of the form $\mathrm{dy} / \mathrm{dt}=\mathrm{ay}+\mathrm{b}$ whose solutions diverge from $\mathrm{y}=2$.
}
\sol \\
Given that the equation is dependent on just $y$, we can say that it is an autonomous differential equation (DE).
We can subsequently solve for the DE by setting $\frac{dy}{dt} = f(y) = 0$, such that its solution is its equilibrium solution.
Since $y=2$, $\boxed{\frac{dy}{dt} = y-2}$. Thus, any number that is less than 2 will result in the solution's phase portrait diverging (-) from $y=2$ --- vice versa for numbers greater than 2.
\subsection{tfi}
\subsection{Classification on DEs}
The order of a DE is the order of the highest derivative that appears in the equation.
\dfn{Linear ODE}{The ODE $F\left(t, y, y^{\prime}, \ldots, y^{(n)}\right)=0$ is said to be linear if $F$ is a linear function of the variables $y, y^{\prime}, \ldots, y^{(n)}$. Thus the general linear ODE of order $n$ is $a_0(t) y^{(n)}+a_1(t) y^{(n-1)}+\ldots+a_n(t) y=g(t)$.}
\qs{pg. 22}{
  Determine the order of the given DE; also state whether the equation is linear or nonlinear: \\
  1) $t^2 y^{\prime \prime}+t y^{\prime}+2 y=\sin t$ \\
  2) $\left(1+y^2\right) y^{\prime \prime}+t y^{\prime}+y=e^t$
}
\sol \\ 
1) The order of the DE is 2, and it is linear: noticed how all the coefficents of n derivatives of $y$ are functions of $t$. \\
2) The order of the DE is 2, and it is nonlinear: noticed how the coefficents of n derivatives of $y$ are functions of $t$ and $y$.

\dfn{Nth-order IDE solutions}{Any function $h$, defined on an interval and possessing at least $n$ derivatives that are continuous on this interval, which substituted into an nth-order ODE reduces the equation to an identity, is said to be a solution of the equation on the interval.}
\qs{pg. 22: \#9)}{
  Verify that the functions $y_1(t)=t^{-2}$ and $y_2(t)=t^{-2} \ln t$ are solutions of the DE
$$ t^2 y^{\prime \prime}+5 t y^{\prime}+4 y=0 $$
}
\sol \\
We can verify that $y_1(t)$ and $y_2(t)$ are solutions of the DE by substituting them into the DE and verifying that the equation holds true. \\
\begin{align*}
  y_1 &= t^{-2} \\
  y_1^{\prime} &= -2t^{-3} \\
  y_1^{\prime \prime} &= 6t^{-4} \\
\end{align*}
Substituting: $t^2 y^{\prime \prime}+5 t y^{\prime}+4 y  = t^2(6t^{-4})+5t(-2t^{-3})+4t^{-2} = 0$ \\
For homework 1, we can verify that $y_2(t)$ is a solution of the DE.

\qs{\#12}{
  Determine the values of $r$ for which $y^{\prime \prime}+y^{\prime}-6 y=0$ has solutions of the form $y=e^{r t}$
}
\sol \\
We can solve for the values of $r$ by substituting $y=e^{rt}$ into the DE and solving for $r$.
$y=e^{rt}$, $y^{\prime}=re^{rt}$, and $y^{\prime \prime}=r^2e^{rt}$. Substituting: $r^2e^{rt}+re^{rt}-6e^{rt}=0$.
Factoring out $e^{rt}$, we get $e^{rt}(r^2+r-6)=0$. Thus, $r^2+r-6=0$, and $\boxed{r=2, -3}$.

\qs{\#14} {
  Determine the values of $r$ for which $t^2 y^{\prime \prime}+4 t y^{\prime}+2 y=0$ has solutions of the form $y=t^r$ for $t>0$
}
\sol \\
$y=t^r$, $y^{\prime}=rt^{r-1}$, and $y^{\prime \prime}=r(r-1)t^{r-2}$.
Substituting: $t^2(r(r-1)t^{r-2})+4t(rt^{r-1})+2t^r=0$.
Factoring out $t^r$, we get $t^r[r(r-1)+4r+2]=0$. $t > 0$ such that $t^r \neq 0$, and $r(r-1)+4r+2=0$. Thus, $r^2+3r+2=0$, and $\boxed{r=-1, -2}$.
\section{tfi}
\subsection{Linear DEs}
\dfn{First-order linear DE}{
  An ODE of the form $$\frac{dy}{dt}+P(t) y= Q(t)$$ is called a first-order linear differential equation in standard form.
  $$ I(x) = e^{\int P(t) dt} $$
  such that,
  $$ y = \frac{1}{I(t)} \int I(t) Q(t) dt + C $$
}
\qs{pg. 31: \#6)}{
  Find the general solution of the differential equation: $t y^{\prime}-\mathrm{y}=\mathrm{t}^2 \mathrm{e}^{-\mathrm{t}}$, where $\mathrm{t}>0$
}
\sol \\
Rewrite the DE: $t \frac{dy}{dt} - y = t^2 e^{-t}$.
Divide both sides by $t$, such that $\frac{dy}{dt} - \frac{1}{t}y = t e^{-t}$.
We can then identify $p(t) = -\frac{1}{t}$, such that $e^{\int p(t) dt} = e^{\int -\frac{1}{t} dt} = e^{-\ln|t|} = e^{ln(t^{-1})} = \frac{1}{t}$.
Therefore, $\frac{d}{dt}[\frac{1}{t}y] = \frac{1}{t} \cdot t e^{-t} = e^{-t}$.
Integrating both sides, $\int \frac{d}{dt}[\frac{1}{t}y] = \int e^{-t} \implies \frac{1}{t}y = -e^{-t} + C \implies \boxed{y=-te^{-t} + Ct}$

\qs{\#11)}{
  Find the solution of the initial value problem: $y^{\prime}+\frac{2}{t} y=\frac{\cos t}{t^2}, y(\pi)=0$, where $\mathrm{t}>0$
}
\sol \\
We know that $p(t) = \frac{2}{t}$, such that $\mu = e^{\int \frac{2}{t} dt} = e^{2\ln|t|} = t^2$.
Multiplying both sides by $\mu$, we get $\frac{d}{dt}[t^2 y] = t^2 \frac{\cos t}{t^2} = \cos t$.
Integrating both sides, we get $\int \frac{d}{dt} [t^2 y] = \int \cos t \implies t^2 y = \sin t + C$.
Such that, $y=\frac{\sin t}{t^2} + \frac{C}{t^2}$. Given that $y(\pi) = 0$, we can solve for $C$.
$0 = \frac{\sin \pi}{\pi^2} + \frac{C}{\pi^2} \implies C = 0$. Therefore, $\boxed{y=\frac{\sin t}{t^2}}$.

\subsection{Separable DEs}
\dfn{Seperable DE}{
  An ODE written in the differential form $M(x) d x+N(y) d y=0$ is said to be separable since terms involving each variable may be placed on opposite sides of the equation. A separable equation can be solved by integrating the functions $M$ and $N$.
}
\qs{pg. 38: \#4)}{
  Solve the DE:$\quad x y^{\prime}=\left(1-y^2\right)^{1 / 2}$
}
\sol \\
We can rewrite the DE as $x \frac{dy}{dx} = \sqrt{1-y^2}$,
we can then seperate the variables and integrate both sides.
$ x dy = \sqrt{1-y^2} dx \implies \int dy \frac{1}{\sqrt{1-y^2}} = \int dx \frac{1}{x}$.
Such that, $\arcsin(y) = \ln|x|+C$. Thus, $\boxed{y=\sin(\ln|x|+C)}$, we call this the general solution.

\dfn{General Order n-th ODE}{
  Trying to solve the ODE $\frac{d^n y}{d x^n}=f\left(x, y, y^{\prime}, \ldots, y^{(n-1)}\right)$ subject to the conditions $\quad y\left(x_0\right)=y_0, y^{\prime}\left(x_0\right)=y_1, \ldots, y^{(n-1)}\left(x_0\right)=y_{n-1}$, where $\mathrm{y}_0, \mathrm{y}_1, \ldots, \mathrm{y}_{\mathrm{n}-1}$ are arbitrary real constants, is called an nth-order initial-value problem $(I V P)$. The values of $\mathrm{y}(\mathrm{x})$ and its first $\mathrm{n}-1$ derivatives at $\mathrm{x}_0$ are called initial conditions (IC).
}

\qs{pg. 38: \#16)}{
  Find the solution of the initial value problem $\sin (2 x) d x+\cos (3 y) d y=0, \quad y\left(\frac{\pi}{2}\right)=\frac{\pi}{3}$ in explicit form.
}
\sol \\
We can solve the DE by seperating the variables and integrating both sides.
$\sin(2x)dx = -\cos(3y)dy \implies \int \sin(2x)dx = -\int \cos(3y)dy$.
Such that, $-\frac{1}{2}\cos(2x) = -\frac{1}{3}\sin(3y) + C$.
We can solve for $C$ by using the initial condition $y(\frac{\pi}{2}) = \frac{\pi}{3}$.
$-\frac{1}{2}\cos(\pi) = -\frac{1}{3}\sin(\pi) + C \implies C = \frac{1}{2}$.
Thus, $-\frac{1}{2} \cos(2x) = \frac{1}{3}\sin(3y) + \frac{1}{2}$.
\begin{align*}
  6[-\frac{1}{2} \cos(2x)] &= 6[-\frac{1}{3}\sin(3y) + \frac{1}{2}] \\
  -3\cos(2x) &= 2\sin(3y) + 3 \\
  -3-3cos(2x) &= -2\sin(3y) \\
  (3\cos(2x)+3) \cdot \frac{1}{2} &= \sin(3y) \\
  \Aboxed{y &= \frac{1}{3} \cdot \arcsin(\frac{3+3\cos(2x)}{2})}
\end{align*}

\dfn{Homogeneous DEs}{
  A first-order DE in differential form $M(x, y) d x+N(x, y) d y=0$ is said to be homogeneous if both coefficient functions $M$ and $N$ have the same degree.
}
\noindent
Either of the substitutions $y=u x$ or $x=v y$, where $u$ and $v$ are new dependent variables, will reduce a homogeneous equation to a separable first-order differential equation.
\double
Although either of the indicated substitutions can be used for every homogeneous differential equation, in practice we try $x=v y$ whenever the function $M$ is simpler than $N$, and $y=u x$ whenever $N$ is simpler than $M$.


\qs{pg. 39: \#26)}{
  Solve: $$\frac{d y}{d x}=\frac{x^2+x y+y^2}{x^2}$$
}
\sol \\
Seperating the differential, $x^2 dy = (x^2+xy+y^2)dx \implies 0 = (x^2 + xy+ y^2) dx - x^2 dy$.
Let $y=ux$, such that $dy = udx + xdu$. Where $0 = (x^2 + x(ux) + (ux)^2)dx - x^2(udx + xdu)$.
Distributing,
\begin{align*}
0 &= x^2 dx + ux^2 dx + u^2x^2 dx - ux^2dx - x^3 du \\
x^3 du &= x^2 dx + u^2x^2 dx \\
\int \frac{du}{u^2+1} &= \int \frac{dx}{x} \\
\arctan(u) &= \ln|x| + C \\
\arctan(\frac{y}{x}) &= \ln|x| + C \\
\Aboxed{y &= x\tan(\ln|x|+C)}
\end{align*}

\qs{\#28) }{
  Solve: $$\frac{d y}{d x}=\frac{4 y-3 x}{2 x-y}$$
}
\sol \\
Cross multplying, $(2x-y) dy = (4y-3x)dx \implies 0 = (4y-3x)dx + (y-2x)dy$.
Let $x=vy$, such that $dx = vdy + ydv$. Where $0 = (4y-3vy)(ydy+ydv) + (y-2vy)dy$.
Distributing,
\begin{align*}
  0 &= 4vydy+4y^2dv-3v^2ydy-3vy^2dv+ydy-2vydy \\
  0 &= 2vydy + 4y^2dv-3v^2ydy-3vy^2dv+ydy \\
  3vy^2dv-4y^2dv &= 2vydy-3vy^2dy-ydy \\
  y^2(3v-4)dv &= y(2v-3v^2+1)dy \\ 
  y^2(3v-4)dv &= -y(3v^2-2v-1) dy \\
  \int \frac{3v-4}{(v-1)(3v+1)} dv &= -\int \frac{1}{y} dy \\
\end{align*}
Breaking up the L.H.S. into partial fractions,
\begin{align*}
  \frac{3v-4}{(v-1)(3v+1)} &= \frac{A}{v-1} + \frac{B}{3v+1} \\
  3v-4 &= A(3v+1) + B(v-1) \\
\end{align*}
Let $v = -\frac{1}{3}$, such that $3(-\frac{1}{3})-4=A(0)+B(-\frac{1}{3}-1) \implies B = \frac{15}{4}$. \\
Let $v = 1$, such that $3(1)-4=A(3+1)+B(0) \implies A = -\frac{1}{4}$. \\
\begin{align*}
  \int \frac{-\frac{1}{4}}{v-1} + \frac{\frac{15}{4}}{3v+1} dv &= -\int \frac{1}{y} dy \\
  -\frac{1}{4} \ln|v-1| + \frac{15}{12} \ln|3v+1| &= -\ln|y| + C \\
  -\frac{1}{3} \ln|\frac{x}{y} - 1| + \frac{5}{4} \ln|\frac{3x}{y}+1| &= -\ln|y| + C \\
  \Aboxed{-\ln|\frac{x-y}{y}| + 5 \ln|\frac{3x+y}{y}| &= -4\ln|y| + C} \\
\end{align*}

\subsection{Modeling with First Order DEs}

\dfn{Initial Value Problem (IVP)}{
  The IVP $\quad \frac{dx}{dt} = kx, \quad x\left(t_0\right)=x_0 \quad$ where $k$ is a constant of proportionality, serves as a model for diverse phenomena involving either growth or decay. The constant $k$ is called the growth or decay constant.
}

\begin{align*}
  \frac{dx}{dt} &= kx \\
  \frac{1}{x} dx &= k dt \\
  \int \frac{1}{x} dx &= \int k dt \\
  \ln|x| &= kt + C \\
  |x| &= e^{kt}e^C = Ce^{kt} \\
  x &= \pm Ce^{kt} = Ce^{kt} \\
  x_0 &= Ce^{k(0)} = C \\
  \Aboxed{x &= x_0 e^{kt}}
\end{align*}

\qs{IVP/Exponential}{
  A culture initially has $P_0$ number of bacteria. At $t=1$ hour the number of bacteria is measured to be $1.5 \mathrm{P}_0$. If the rate of growth is proportional to the number of bacteria $P(t)$ present at time $t$, determine the time necessary for the number of bacteria to triple.
}
\sol \\
$\frac{dP}{dt} = kP$, where $P(0) = P_0$, $P(1) = 1.5P_0$. We want to solve for $t$ such that $P(t) = 3P_0$.
We know that $P(t) = P_0 e^{kt}$, such that $1.5P_0 = P_0 e^{k(1)} \implies 1.5 = e^k \implies ln(1.5) = k \approx 0.4055$.
Such that, $P = P_0 e^{0.4055t} \implies 3P_0 = P_0 e^{0.4055t} \implies 3 = e^{0.4055t} \implies ln(3) = 0.4055t \implies t = \frac{\ln 3}{0.4055} = \boxed{2.7 \text{hours.}}$

\dfn{Linear First-Order DEs}{
  The mixing of two fluids sometimes gives rise to a linear first-order differential equation. When two brine solutions are mixed, if we assume that the rate $A^{\prime}(t)$ at which the amount of salt in the mixing tank changes is a net rate, then
$$
\begin{aligned}
& \mathrm{dA} / \mathrm{dt}=\text { (input rate of salt) }- \text { (output rate of salt) } \\
& =\quad R_{\text {in }} \quad-\quad R_{\text {out }} \\
&
\end{aligned}
$$
}

\qs{Linear FODE}{
  Suppose that a large mixing tank initially holds 300 gallons of brine. Another brine solution is pumped into the large tank at a rate of 3 gallons/minute, and the concentration of the salt in this inflow is 2 pounds/gallon. When the solution in the tank is well stirred, it is pumped out at the same rate as the entering solution. If 50 pounds of salt were dissolved initially in the 300 gallons, how much salt is in the tank after a long time?
  \imgg{qs2.8}{0.4}
}
\sol \\
We can model the ROC of the salt in the tank as $\frac{dA}{dt} = R_{in} - R_{out}$,
where $\frac{dA}{dt}$ is modeled as lbs/min (such that both in and out have to be the same units).
Such that, $R_{in} = (3 \frac{gal}{min})(2 \frac{lbs}{gal}) = 6 \frac{lbs}{min}$.
We can then model $R_{out} = (3 \frac{gal}{min})(\frac{A}{300} \frac{lbs}{gal}) = \frac{A}{100} \frac{lbs}{min}$.
Thus, $\frac{dA}{dt} = 6 - \frac{A}{100}$, where $A(0) = 50$.
We can rewrite the DE as $\frac{dA}{dt} + \frac{1}{100}A = 6$, such that $p(t) = \frac{1}{100}$ and $g(t) = 6$.
Where $\mu = e^{\int \frac{1}{100} dt} = e^{\frac{1}{100}t}$.
Such that, $\frac{d}{dt} [e^{\frac{1}{100}t}A] = 6e^{\frac{1}{100}t} \implies \int \frac{d}{dt} [e^{\frac{1}{100}t}A] = \int 6e^{\frac{1}{100}t}
\implies e^{\frac{1}{100}t} A = 6 \cdot 100 e^{\frac{1}{100}t} + C \implies A = 600 + \frac{C}{e^{\frac{1}{100}t}}$.
Given that $A(0) = 50$, we can solve for $C$ such that $50 = 600 + \frac{C}{e^0} \implies C = -550$.
Therefore, $A=600-\frac{550}{e^{\frac{1}{100}t}}$, where after a long time, $\lim_{t \to \infty} A = \boxed{600 \text{ lbs}}$.

\dfn{Newton's Law of Cooling}{
  $$ \frac{d T}{d t}=k\left(T-T_m\right) $$
  where $k$ is the constant of proportionality, $\mathrm{T}(\mathrm{t})$ is the temperature of the object for $t>0$, and $T_m$ is the temperature of the medium surrounding the object.
}

\qs{Newton's Law of cooling}{
  When a cake is removed from an oven, its temperature is measured at 300 degrees F.
  Three minutes later its temperature is 200 degrees F. How long will it take to cool to a temperature of 75 degrees $\mathrm{F}$,
  if the room temperature is 70 degrees $\mathrm{F}$?
}
\sol \\
We can set up a DE, $\frac{dT}{dt} = k(T-70)$, where $T(0) = 300$ and $T(3) = 200$.
Seperating the variables, $\frac{dT}{T-70} = k dt \implies \int \frac{dT}{T-75} = \int k dt
\implies \ln|T-70| = kt + C$.
Where at $T(0) = 300 \implies \ln 230 = C$.
Also at $T(3) = 200 \implies \ln 130 = 3k + \ln 230 \implies k = \frac{\ln 130 - \ln 230}{3} \approx -0.19018$.
Raising both by $e$,
\begin{align*}
  e^{ln|T-70|} &= e^{-0.19018t + \ln 230} \\
  T-70 &= 230 e^{-0.19018t} \\
  T &= 70 + 230 e^{-0.19018t} \\
  75 &= 70 + 230 e^{-0.19018t} \\
  5 &= 230 e^{-0.19018t} \\
  \frac{5}{230} &= e^{-0.19018t} \\
  ln (\frac{1}{46}) &= ln(e^{-0.19018t}) \\
  t &= \frac{\ln(\frac{1}{46})}{-0.19018} \approx \boxed{20.1 \text{ minutes}}
\end{align*}

\subsection{Exact DEs}

\dfn{Exact Differential}{
  A differential expression $M(x, y) d x+N(x, y) d y=0$ is an exact differential if it corresponds to the differential of some function $f(x, y)$. A first-order differential equation of the form
  $$ M(x, y) d x+N(x, y) d y=0 $$
  is said to be an exact equation if the expression on the left-hand side is an exact differential.
  \double
  A necessary and sufficient condition that $M(x, y) d x+N(x, y) d y$ be an exact differential is $\frac{\partial M}{\partial y}=\frac{\partial N}{\partial x}$.
}

\qs{pg. 75: \#6)}{Determine whether the equation
$$ \left(y e^{x y} \cos (2 x)-2 e^{x y} \sin (2 x)+2 x\right)+\left(x e^{x y} \cos (2 x)-3\right) y^{\prime}=0 $$
is exact. If it is exact, find the solution.
}
\sol \\
$(y e^{xy} \cos (2x) - 2 e^{xy} \sin (2x) +2x) dx + (x e^{xy} \cos (2x) - 3) dy = 0$, where $M$ is the first term and $N$ is the second term.
\begin{align*}
  \del{M}{y} &= e^{xy} \cos(2x) + yx e^{xy} \cos(2x) - 2xe^{xy} \sin(2x) \\
  \del{N}{x} &= e^{xy} \cos(2x) + yx e^{xy} \cos(2x) - 2xe^{xy} \sin(2x)
\end{align*}
Therefore they are exact such that $M = \del{f}{x}$ and $N = \del{f}{y}$.
\begin{align*}
  \int \del{f}{y} &= \int (x e^{xy} \cos (2x) -3) dy \\
  f(x,y) &= x \cos (2x) \frac{e^{xy}}{x} -3y + g(x) \\
  \del{f}{x} &= ye^{xy} \cos(2x) - 2e^{xy} \sin(2x) + g'(x) \\
  &= ye^{xy} \cos(2x) - 2e^{xy} \sin(2x) + 2x \implies g'(x) = 2x \implies g(x) = x^2 \\
  f(x,y) &= \boxed{e^{xy} \cos(2x)  -3y + x^2 = C}
\end{align*}

\nt{
  $$\frac{d}{dx} (fgh) = f'gh + fg'h + fgh'$$
}

\qs{\#10)}{Solve the initial value problem:
$$\left(9 x^2+y-1\right)-(4 y-x) y^{\prime}=0, \quad y(1)=0 $$
}

\sol \\
$(9x^2+y-1)dx + (x-4y)dy = 0$, $y(1) = 0$, where $M$ is the first term and $N$ is the second term. \\
$\del{M}{y} = 1 = \del{N}{x} \implies$ exact.
\begin{align*}
  \int \del{f}{x} &= \int (9x^2+y-1) dx \\
  f(x,y) &= 3x^3 + xy - x + g(y) \\
  \del{f}{y} &= x + g'(y) = x - 4y \\
  \implies g(y) &= -2y^2 \\
  f(x,y) = C &= 3x^3 + xy - x - 2y^2 \\
  y(1)= 0 \implies 2 &= C \\
  \Aboxed{3x^3 + xy -x -2y^2 &= 2}
\end{align*}

\qs{\#12) }{Find the value of $b$ for which the equation
$$ \left(y e^{2 x y}+x\right)+b^{2 x y} y^{\prime}=0 $$
is exact, and then solve it using that value of $b$.}
\sol \\
$(y e^{2xy} + x)dx + b^{2xy}dy = 0$, where $M$ is the first term and $N$ is the second term. \\
\begin{align*}
  \del{M}{y} &= e^{2xy} + 2xy e^{2xy} \\
  \del{N}{x} &= b [e^{2xy} + 2xy e^{2xy}] \implies b = 1 \\
\end{align*}
Such that the equation is exact when $b=1$.
\begin{align*}
  \int \del{f}{y} &= \int (xe^{2xy}) dy \\
  f(x,y) &= \frac{1}{2}e^{2xy} + g(x) \\
  \del{f}{x} &= \frac{1}{2}(2y e^{2xy}) + g'(x) \\ 
  \del{f}{x} &= y e^{2xy} + g'(x) = y e^{2xy} + x \\
  &\implies g'(x) = x \implies g(x) = \frac{1}{2}x^2 \\
  f(x,y) &= C = \frac{1}{2}e^{2xy} + \frac{1}{2}x^2 \\
  \Aboxed{C &= e^{2xy} + x^2}
\end{align*}

\subsection{Euler's Method}
\dfn{Euler's Method}{
  For small $\Delta t, \frac{d y}{d t} \approx \frac{\Delta y}{\Delta t}$, or equivalently, $\Delta y \approx \frac{d y}{d t} \Delta t$.
}

\qs{pg. 82: \#4}{
  Given $y^{\prime}=3 \cos t-2 y, \quad y(0)=0$. \\ 
a) Find approximate values of the solution of the IVP at $\mathrm{t}=0.1,0.2,0.3$, and 0.4 using Euler's method with $h=\Delta t=0.1$. \\
d) Find the actual solution of the IVP and evaluate the solution at $t=0.1,0.2,0.3$, and 0.4 . Compare these values with the results in part a).
}
\sol a) We have that $\Delta y = \frac{dy}{dt} \Delta t$ such that, $\Delta y = (3 \cos t - 2y) (0.1)$.

\begin{center}
\begin{tabular}{l|l|l}
  $t$ & $y$    & $\Delta y$ \\
  0   & 0      & 0.3        \\
  0.1 & 0.3    & 0.2385     \\
  0.2 & 0.5385 & 0.1863     \\
  0.3 & 0.7248 & 0.1416     \\
  0.4 & 0.8664  &           
\end{tabular}
\end{center}
\noindent
d) Given $\frac{dy}{dt} + 2y  = 3\cos t$, $y(0) =0$.
Such that $I(t) = e^{\int 2 dt} = e^{2t}$.
\begin{align*}
  \frac{d}{dt} [e^{2t}y] &= 3 e^{2t} \cos t \\
  y &= \frac{1}{e^{2t}} \int e^{2t} 3 \cos t dt \\
\end{align*}
Let $u = \cos t \implies du = -\sin t dt$, $dv = 3 e^{2t} \implies v = \frac{3}{2} e^{2t}$
$$ \int e^{2t} 3 \cos t dt = \frac{3}{2} e^{2t} \cos t + \frac{3}{2} \int e^{2t} \sin t dt $$
Let $u = \sin t \implies du = \cos t dt$, $dv = e^{2t} \implies v = \frac{1}{2} e^{2t}$.
\begin{align*}
&= \frac{3}{2} e^{2t} \cos t + \frac{3}{2} \left( \frac{1}{2} e^{2t} \sin t - \frac{1}{2} \int e^{2t} \cos t dt \right) \\
3 \int e^{2t} \cos t dt &= \frac{3}{2} e^{2t} \cos t + \frac{3}{4} e^{2t} \sin t - \frac{3}{4} \int e^{2t} \cos t dt \\
\frac{15}{4} \int e^{2t} \cos t dt &= \frac{3}{2} e^{2t} \cos t + \frac{3}{4} e^{2t} \sin t \\
\int e^{2t} \cos t dt &= \frac{2}{5} e^{2t} \cos t + \frac{1}{5} e^{2t} \sin t \\
y &= \frac{3}{e^{2t}} [\frac{2}{5} e^{2t} \cos t + \frac{1}{5} e^{2t} \sin t] + C \\
y &= \frac{6}{5} \cos t + \frac{3}{5} \sin t + Ce^{-2t} \\
0 &= \frac{6}{5} \cos 0 + \frac{3}{5} \sin 0 + Ce^{-2(0)} \implies C = -\frac{6}{5} \\
\Aboxed{y &= \frac{6}{5} \cos t + \frac{3}{5} \sin t - \frac{6}{5} e^{-2t}}
\end{align*}
Testing $y(0.4) = \frac{6}{5} \cos 0.4 + \frac{3}{5} \sin 0.4 - \frac{6}{5} e^{-2(0.4)} \boxed{\approx 0.7997}$. \\

\qs{\#10}{
  Use Euler's method to find approximate values of the solution of the IVP $\quad \mathrm{y}^{\prime}=\mathrm{y}(3-\mathrm{ty}), \quad \mathrm{y}(0)=0.5, \quad$ at $\mathrm{t}=0.5$ with $h=\Delta t=0.1$.
}
\sol We have that $\Delta y = y(3-ty) (0.1)$
\begin{center}
  \begin{tabular}{l|l|l}
    $t$ & $y$    & $\Delta y$ \\
    0   & 0.5    & 0.15       \\
    0.1 & 0.65   & 0.1908     \\
    0.2 & 0.8408 & 0.2381     \\
    0.3 & 1.0789 & 0.2887     \\
    0.4 & 1.3676 & 0.335      \\
    0.5 & 1.7031 & 
  \end{tabular}
  \end{center}
$\boxed{y(0.5) \approx 1.7031}$.

\subsection*{3.1, 3.2, 3.4, 4.1}
\dfn{Linear Dependence}{
  A set of functions $f_1(t), f_2(t), \ldots, f_n(t)$ is said to be linearly dependent on an interval I if there exists constants $c_1, c_2, \ldots, c_n$ not all zero, such that
  $$ c_1 f_1(t)+c_2 f_2(t)+\ldots+c_n f_n(t)=0 $$
  for every $t$ in the interval. Such that if not, the set is linearly independent.
}

\ex{}{
  If $c_1 f_1(t) + c_2 f_2(t) + ... + c_n f_n (t) = 0 \implies c_1 = c_2 = ... = c_n = 0$
  Therefore the set (and thereby said functions to $n$) are linearly independent.
}

\dfn{Wronskian}{
  Suppose each of the functions $f_1(t), f_2(t), \ldots, f_n(t)$ possesses at least $\mathrm{n}-1$ derivatives. The determinant
  $$W=\left|\begin{array}{cccc}
  f_1 & f_2 & \ldots & f_n \\
  f_1^{\prime} & f_2^{\prime} & \ldots & f_n^{\prime} \\
  \vdots & \vdots & \ddots & \vdots \\
  f_1^{(n-1)} & f_2^{(n-1)} & \ldots & f_n^{(n-1)}
  \end{array}\right| $$
is called the Wronskian of the functions.
}

\dfn{Homogeneous Linear n-th Order ODE}{
  A linear nth-order ODE of the form
  $$ a_n(x) \frac{d^n y}{d x^n}+a_{n-1}(x) \frac{d^{n-1} y}{d x^{n-1}}+\ldots+a_1(x) \frac{d y}{d x}+a_0(x) y=0 $$
  is said to be homogeneous. Note the R.H.S being 0.
}

\thm{Fundamental Set of Solutions}{
  Let $y_1, y_2, \ldots, y_n$ be $\mathrm{n}$ solutions of a homogeneous linear $\mathrm{nth}$-order ODE on the interval I. Then the set of solutions is linearly independent on I if and only if $W \neq 0$ for every $\mathrm{t}$ in the interval.
}

\qs{}{
  Use the Wronskian to prove that the following functions are linearly independent:\\
  a) $y_1(t)=e^{3 t}, y_2(t)=e^{-3 t}$ \\
  b) $\quad y_1(t)=e^t, y_2(t)=e^{2 t}, y_3(t)=e^{3 t}$
}
\sol If we want to prove linearly independence we want a non zero Wronskian. \\ 
a) $W = \left| \begin{array}{cc} e^{3t} & e^{-3t} \\ 3e^{3t} & -3e^{-3t} \end{array} \right| = e^{3t}(-3e^{-3t}) - e^{-3t}(3e^{3t}) = -3e^{3t - 3t} - 3e^{3t - 3t} = -6 \ne 0$. Therefore, the set is linearly independent. $\qed$ \\
b) $W = \left| \begin{array}{ccc}
        e^t & e^2t & e^{3t} \\
        e^t & 2e^2t & 3e^{3t} \\
        e^t & 4e^2t & 9e^{3t}
        \end{array}\right| = 2e^{6t} \ne 0$ Therefore the set is linearly independent. $\qed$ \\

\thm{Fundamental set of solutions on an interval}{
  Any set $y_1, y_2, \ldots, y_n$ of $\mathrm{n}$ linearly independent solutions of a homogeneous linear nth-order ODE on an interval I is said to be a fundamental set of solutions on the interval.
}

\dfn{General Solution of a Homogeneous Linear nth-Order ODE}{
  Let $y_1, y_2, \ldots, y_n$ be a fundamental set of solutions of a homogeneous linear nth-order ODE on an interval I. Then the general solution of the equation on the interval is:
  $$ y=c_1 y_1+c_2 y_2+\ldots+c_n y_n $$
  where $c_1, c_2, \ldots, c_n$ are arbitrary constants.
}

\dfn{Auxiliary Equations}{
  The auxiliary equation to the ODE
  $$a y^{\prime \prime}+b y^{\prime}+c y=0 \text { is } a r^2+b r+c=0 $$
  (Sec. 3.1) If the auxiliary equation has two distinct roots $r_1$ and $r_2$, then the general solution is
  $$y=c_1 e^{r_1 x}+c_2 e^{r_2 x}$$
  (Sec. 3.4) If the auxiliary equation has two equal roots, then the general solution is
  $$y=c_1 e^{r_1 x}+c_2 x e^{r_1 x}$$
}

\ex{}{
  Given $ay'' + by' + cy = 0$, we try
  $y =e^{rt}, y' = re^{rt}, y'' = r^2e^{rt}$. Such that $a(r^2e^{rt}) + b(re^{rt}) + ce^{rt} = 0 \implies e^{rt} (ar^2 +br +c ) = 0$.
  Where we solve such quadratic for $r$.
}

\qs{}{
  Solve the following ODEs: \\
  a) $2 y^{\prime \prime}-5 y^{\prime}-3 y=0$ \\
  b) $y^{\prime \prime}-10 y^{\prime}+25 y=0$
}
\sol
a) The auxiliary equation is $2r^2-5r-3 = 0 = (r-3)(2r+1)$ Such that, $r_1=3, r_2-\frac{1}{2}$.
Therefore the general solution is $\boxed{y= c_1 e^{3t} + c_2 e^{-\frac{1}{2}t}}$. \\
b) The auxiliary equation is $r^2-10r+25 = 0 = (r-5)^2$ Such that, $r_1 = r_2 = 5$. 
Therefore the general solution is $\boxed{y = c_1 e^{5x} + c_2 x e^{5x}}$. \\

\qs{3.1 \#7}{
  Solve the ODE,
  $$ y'' + y' -2y = 0, \quad y(0) = 1, \quad y'(0) = 1 $$
}
\sol The auxiliary equation $r^2 + r -2 = 0 (r-1)(r+2) \implies r_1 = 1, r_2 = -2$.
Such that the general solution $y = c_1 e^{t} + c_2 e^{-2t}$.
\begin{align*}
  1 &= c_1e^0 + c_2e^{-2(0)} = c_1 + c_2 \\
  y' &= c_1e^t - 2c_2e^{-2t} \\
  1 &= c_1e^0 - 2c_2e^{-2(0)} = c_1 - 2c_2 \\
  \intertext{Row reducing...}
  \implies & c_2 = 0, c_1 = 1 \\
  \Aboxed{y &= e^t}
\end{align*}

\qs{3.1 \#13}{
  Find a differential equation whose general solution is $y=c_1 e^{2t}+c_2 e^{-3t}$.
}
\sol We have that $r_1 = 2, r_2 = -3$. Such that the factored form of our equation is $(r-2)(r+3) = 0 \implies r^2 + r - 6 = 0$.
Therefore $\boxed{y'' + y' - 6y = 0}$.

\ex{Repeated roots}{
  \begin{align*}
    y''-2y'+y &=0 \\
    \implies r^2 -2r+1 &= 0 \\
    (r-1)^2 &= 0 \\
    r_1 = r_2 &= 1 \\
    \therefore \Aboxed{y &= c_1 e^t + c_2 t e^t}
  \end{align*}
}

\subsection*{3.3, 4.2}
\thm{Euler's Formula}{
  $$e^{i \theta}=\cos \theta+i \sin \theta$$
}
\proof
\begin{align*}
  &\cos \theta + i \sin \theta \\
  &= (1 - \frac{\theta^2}{2!} + \frac{\theta^4}{4!} - \frac{\theta^6}{6!} + \ldots) + i (\theta - \frac{\theta^3}{3!} + \frac{\theta^5}{5!} - \frac{\theta^7}{7!} + \ldots) \\
  &= 1 - \frac{\theta^2}{2!} + \frac{\theta^4}{4!} - \frac{\theta^6}{6!} + \ldots + i \theta - i \frac{\theta^3}{3!} + i \frac{\theta^5}{5!} - i \frac{\theta^7}{7!} + \ldots \\
  \intertext{Rearanging by degree...}
  &= 1 + i \theta - \frac{\theta^2}{2!} - i \frac{\theta^3}{3!} + \frac{\theta^4}{4!} + i \frac{\theta^5}{5!} - \frac{\theta^6}{6!} - i \frac{\theta^7}{7!} + \ldots
  \intertext{Going back to the R.H.S}
  e^{i \theta} &= 1 + i \theta - \frac{\theta^2}{2!} - i \frac{\theta^3}{3!} + \ldots \\
  &= 1 + i \theta \frac{i^2 \theta^2}{2!} + \frac{i^3 \theta^3}{3!} + \ldots = 1 + i \theta - \frac{\theta^2}{2!} - \frac{i\theta^2}{2!} - \frac{\theta^3}{3!} + \ldots \\
  \intertext{Such that $i$ keeps "oscillating" between the real and imaginary parts.}
  &= \cos \theta + i \sin \theta \qed
\end{align*}
\dfn{Complex Auxiliary Equations}{
  If the solutions of the auxiliary equation for $a y^{\prime \prime}+b y^{\prime}+c y=0$ are complex, $r=\alpha \pm \beta i$, then the general solution is
  $$ y=e^{\alpha x}\left(c_1 \cos \beta x+c_2 \sin \beta x\right)$$
}
\proof
Given that two $r$ solutions to the auxiliary equation
\begin{align*}
  r_1 &= \alpha + \beta i, r_2 = \alpha - \beta i \\
  \intertext{The general solution is}
  y &= c_1 e^{(\alpha + \beta i)x} + c_2 e^{(\alpha - \beta i)x} \\
  &= c_1 e^{\alpha x} e^{i \beta x} + c_2 e^{\alpha x} e^{-i \beta x} \\
  \intertext{Such that the following equals}
  e^{\beta x i} &= \cos \beta x + i \sin \beta x \\
  e^{-\beta x i} &= \cos \beta x - i \sin \beta x \\
  \implies e^{i \beta x} + e^{-i \beta x} &= 2 \cos \beta x \\
  \implies e^{i \beta x} - e^{-i \beta x} &= 2i \sin \beta x
  \intertext{Let $c_1 = c_2 = 1$ such that we can solve for a specific solution $y_1$}
  y_1 &= e^{\alpha x} e^{\beta x i} + e^{\alpha x} e^{-\beta x i} \\
  &= e^{\alpha x} (e^{\beta x i} + e^{-\beta x i}) \\
  &= 2e^{\alpha x} (\cos \beta x) \\
  \intertext{Let $c_1 = 1, c_2 -1$ such that we can solve for a specific solution $y_2$}
  y_2 &= e^{\alpha x} (e^{\beta x i} - e^{-\beta x i}) \\
  &= 2i e^{\alpha x} (\sin \beta x)
  \intertext{Such that we can form another solution by adding particular solutions}
  y_1 + y_2 &= 2e^{\alpha x} (\cos \beta x) + 2i e^{\alpha x} (\sin \beta x)
  \intertext{Where subsequently the solution is true for another constant, such that the general solution is}
  \Aboxed{y &= e^{\alpha x} (c_1 \cos \beta x + c_2 \sin \beta x)}
\end{align*}
\qs{}{
  Solve the ODE
  $$ y^{\prime \prime}+4 y^{\prime}+7 y=0 $$
}
\sol The auxiliary equation is $r^2 + 4r + 7 = 0$.
\begin{align*}
  r &= \frac{-4 \pm \sqrt{4^2 - 4(1)(7)}}{2} \\
  &= \frac{-4 \pm \sqrt{-12}}{2} = \frac{-4 \pm \sqrt{-4 \cdot 3}}{2} = \frac{-4 \pm 2i\sqrt{3}}{2} = -2 \pm i\sqrt{3} \\
  &\implies \alpha = -2, \beta = \sqrt{3} \\
  \Aboxed{y &= e^{-2x} (c_1 \cos \sqrt{3}x + c_2 \sin \sqrt{3}x)}
\end{align*}

\qs{}{
  Solve the initial-value problem
  $$4 y^{\prime \prime}+4 y^{\prime}+17 y=0, \quad y(0)=-1, \quad y^{\prime}(0)=2$$
}
\sol The auxiliary equation is $4r^2 + 4r + 17 = 0$.
\begin{align*}
  r &= \frac{-4 \pm \sqrt{4^2 - 4(4)(17)}}{8} = \frac{-4 \pm \sqrt{-4 \cdot 3}}{8} \\
  &= \frac{-4 \pm 16i}{8} = -\frac{1}{2} \pm 2i \\
  &\implies \alpha = -\frac{1}{2}, \beta = 2 
  \intertext{Such that the general solution is}
  y &= e^{-\frac{1}{2}x} (c_1 \cos 2x + c_2 \sin 2x)
  \intertext{Given that $y(0) = -1$}
  -1 &= e^{0} (c_1 \cos 0 + c_2 \sin 0) \implies c_1 = -1 \\
  \implies y &= e^{-\frac{1}{2}x} (-\cos 2x + c_2 \sin 2x)
  \intertext{Taking the derivative (chain)...}
  y' &= -\frac{1}{2}e^{\frac{1}{2}x} (-\cos 2x + c_2 \sin 2x) + e^{-\frac{1}{2}x} (2 \sin x + 2c_2 \cos 2x)
  \intertext{Given that $y'(0) = 2$}
  2 &= -\frac{1}{2}e^{0} (-\cos 0 + c_2 \sin 0) + e^{0} (2 \sin 0 + 2c_2 \cos 0) \\
  2 &= -\frac{1}{2} (-1) + 2c_2 \implies c_2 = \frac{3}{4}
  \intertext{Therefore the particular solution is}
  \Aboxed{y &= e^{-\frac{1}{2}x} (-\cos 2x + \frac{3}{4} \sin 2x)}
\end{align*}

\qs{}{
  Solve the DE  
  $$y^{\prime \prime \prime}+3 y^{\prime \prime}-4 y=0$$
}
\sol The auxiliary equation is $r^3 + 3r^2 - 4 = 0$.
We can say that are possible rational solutions are all factors of $p=-4$ and $p=1$, $\frac{p}{q} = \pm 1, \pm 2, \pm 4$.
By inspection, $r=1$ is a solution. Therefore, we can factor the polynomial by synthetic division.
\begin{align*}
  (r-1)(r^2 + 4r + 4) &= 0 \\
  (r-1)(r+2)^2 &= 0 \\
  r_1 = 1, r_2 = -2, r_3 = -2
  \intertext{Therefore the general solution is}
  \Aboxed{y = c_1 e^x + c_2 e^{-2x} + c_3 x e^{-2x}}
\end{align*}
Note that we can graph the function (calculator lol) to find the solutions ($r$). Where through points are definite and touch points are reflective of two solutions (such as $r_2 = r_3 = -2$). \\

\ex{No $y'$ term with imaginary term}{
  Given $y'' + k^2 y =0$.
  We can say the auxiliary equation is $r^2 + k^2 = 0 \implies r = \pm ki \implies r = 0 \pm ki$.
  Where $\alpha = 0$ and $\beta = k$. Therefore the general solution is
  $$y = c_1 \cos kx + c_2 \sin kx$$
}

\ex{No $y'$ term with real term}{
  Given $y'' - k^2 y =0$.
  We can say the auxiliary equation is $r^2 - k^2 = 0 \implies r = \pm k$.
  Therefore the general solution is
  $$y = c_1 e^{kx} + c_2 e^{-kx}$$
  Note that we can also find a specific solution such that
  $$c_1 = c_2 = \frac{1}{2} \implies y_1 = \frac{e^{kx}+e^{-kx}}{2} = \cosh kx$$
  Also where,
  $$c_1 = \frac{1}{2}, c_2 = -\frac{1}{2} \implies y_2 = \frac{e^{kx}-e^{-kx}}{2} = \sinh kx$$
  Where we can combine linear combinations of these to form the general solution.
  $$ y = c_1 \cosh kx + c_2 \sinh kx $$
}

\qs{}{
  Solve a)
  $$ y'' + 3y = 0 $$
  b) $$y'' -9y =0$$
}
\sol \\
a)
$$ y = c_1 \cos \sqrt{3}x + c_2 \sin \sqrt{3}x $$
b) 
$$ y = c_1 \cosh 3x + c_2 \sinh 3x \quad \text{or} \quad y = c_1 e^{3x} + c_2 e^{-3x}$$

\qs{}{
  Solve the following ODE
  $$\frac{d^4 y}{d t^4}+2 \frac{d^2 y}{d t^2}+y=0$$
}
\sol The auxiliary equation is $r^4 + 2r^2 + 1 = 0$.
We can factor the equation such that $\left(r^2+1 \right)^2 = 0 \implies r^2 = 1 \implies r = \pm i$.
Subsequenty that $r = 0 \pm i$ such that $\alpha = 0, \beta = 1$.
Therefore the general solution is
$$\boxed{y = c_1 \cos t + c_2 \sin t + c_3 t \cos t + c_4 t \sin t}$$
Note that the OG equation is to the 4th degree, such that we have 4 solutions.

\subsection*{3.5 Nonhomogeneous Equations; Method of Undetermined Coefficients}
\dfn{Second Order Linear Nonhomogeneous DE}{
  A second-order linear nonhomogeneous DE is an equation of the form
  $$L[y]=y^{\prime \prime}+p(t) y^{\prime}+q(t) y=g(t)$$
  where $g(t)$ is a continuous function on an interval I, such that it is a polynomial, exponential, sine, cosine, or a linear combination of these.
  \double
  In a nonhomogeneous ODE, the solution to the corresponding homogeneous ODE is called the complementary solution and is denoted by $\mathrm{y}_{\mathrm{c}}$.
  \double
  A particular solution to a nonhomogeneous ODE is denoted by $y_p$.
  \double
  Superposition Principle: The general solution to a nonhomogeneous $O D E$ is $y=y_c+y_p$.
}

\qs{}{
  Solve
  $$ y^{\prime \prime}+3 y^{\prime}+2 y=4 x^2 $$
}
\sol \\
We first solve the homogeneous equation $y^{\prime \prime}+3 y^{\prime}+2 y=0$.
The auxiliary equation is $r^2 + 3r +2 = 0 \implies (r+2)(r+1)$, such that $r_1 = -2, r_2 = -1$.
Therefore the complementary solution is $y_c = c_1 e^{-2x} + c_2 e^{-x}$.
\double
We subsequently guess a particular solution $y_p$ (such that we begin with a general 2nd degree polynomial).
Where $y_p = Ax^2 + Bx + C \implies y_p' = 2Ax + B \implies y_p'' = 2A$.
Such that,
\begin{align*}
y_p'' + 3y_p' + 2y_p &= 2A + 3(2Ax + B) + 2(Ax^2 + Bx + C) \\
&= 2A + 6Ax + 3B + 2Ax^2 + 2Bx + 2C = 4x^2 \\
x^2 &: 2A = 4 \implies A = 2 \\
x &: 6A + 2B = 0 \implies 12 + 2B = 0 \implies B = -6 \\
\text{Constant} &: 2A + 3B +2C = 0 \implies 4 -18 + 2C = 0 \implies C = 7
\end{align*}
Therefore the particular solution is $y_p = 2x^2 - 6x + 7$.
Forming the general solution,
$$\boxed{y = y_c + y_p = c_1 e^{-2x} + c_2 e^{-x} + 2x^2 - 6x + 7}$$

\qs{}{
  Solve  
  $$ y^{\prime \prime}-3 y^{\prime}=8 e^{3 x}+4 \sin x $$
}
\sol \\
We first solve the homogeneous equation $y^{\prime \prime}-3 y^{\prime}=0$.
The auxiliary equation is $r^2 - 3r = 0 \implies r(r-3) = 0 \implies r_1 = 0, r_2 = 3$.
Therefore the complementary solution is $y_c = c_1 + c_2 e^{3x}$.
\double
Since we have $c_2 e^{3x}$ in the complementary solution,
we must guess a particular solution such that $y_p = Axe^{3x} + B \cos x + C \sin x$.
$y_p' = A(e^{3x} + 3x e^{3x}) - B \sin x + C \cos x$ and $y_p'' = A(6e^{3x} + 9x e^{3x}) - B \cos x - C \sin x$.
Such that,
\begin{align*}
  y_p'' - 3y_p' &= A(6e^{3x} + 9x e^{3x}) - B \cos x - C \sin x - 3[A(e^{3x} + 3x e^{3x}) - B \sin x + C \cos x] \\
  &= 3A e^{3x} - B \cos x - C \sin x + 3B \sin x - 3C \cos x = 8e^{3x} + 4 \sin x \\
  e^{3x} &: 3A = 8 \implies A = \frac{8}{3} \\
  \cos x &: -B -3C = 0 \\
  \sin x &: 3B + 4C = 4 \implies C = -\frac{2}{5} \implies B = \frac{6}{5}
\end{align*}
Therefore the particular solution is $y_p = \frac{8}{3}xe^{3x} + \frac{6}{5} \cos x - \frac{2}{5} \sin x$.
Forming the general solution,
$$\boxed{y = y_c + y_p = c_1 + c_2 e^{3x} + \frac{8}{3}xe^{3x} + \frac{6}{5} \cos x - \frac{2}{5} \sin x}$$

\qs{}{
  Solve  
  $$ y^{\prime \prime}+y=x \cos x-\cos x $$
}
\sol The homogeneous equation is $y^{\prime \prime}+y=0$ such that the auxiliary (characteristic) equation is $r^2 + 1 = 0 \implies r^2 = -1 \implies r = 0 \pm i$.
Therefore the complementary solution is $y_c = c_1 \cos x + c_2 \sin x$.
\double
We already have a $\cos x$ in the complementary solution (where $c_1 \cos x + c_2 \sin x$), such that we must guess a particular solution such that for $\cos x$, $y_p = (Ax + B) \cos x + (Cx + D) \sin x$.
Such that, $y_p'' = \left(-Ax^2 + (-B+4C)x + (2A+2D)\right) \cos x + (-Cx^2 + (-4A-D)x -2B-2C) \sin x $.
\begin{align*}
  y_p'' + y_p &= \left(-Ax^2 + (-B+4C)x + (2A+2D)\right) \cos x + (-Cx^2 + x -2B+2C) \sin x \\
  &+ (Ax + B) \cos x + (Cx + D) \sin x
  \intertext{Simplifying the L.H.S.}
  &= (4Cx + (2A+2D))\cos x + (-4Ax + (x-2B+2C))\sin x = x \cos x - \cos x
  \intertext{Solving for the coefficents}
  & -4A = 0 \implies A = 0 \\
  & -2B-2C = 0 \implies B = C \\
  & 4C = 1 \implies C = \frac{1}{4} = B \\
  & 2D = -1 \implies D = -\frac{1}{2}
\end{align*}
Therefore the particular solution is $y_p = \frac{1}{4}x \cos x + \frac{1}{4} x^2 \sin x - \frac{1}{2} \sin x$.
Forming the general solution,
$$\boxed{y = y_c + y_p = c_1 \cos x + c_2 \sin x + \frac{1}{4}x \cos x + \frac{1}{4} x^2 \sin x - \frac{1}{2} x \sin x }$$

\qs{}{
  Determine the form of the particular solution for the following nonhomogeneous ODE:
  $$ y^{\prime \prime}-2 y^{\prime}+y=10 e^{-2 x} \cos x $$
}
\sol The auxiliary equation is $r^2 - 2r + 1 =0 \implies (r-1)^2 = 0 \implies r = 1$.
Such that the complementary solution is $y_c = c_1 e^x + c_2 x e^x$.
Therefore $\boxed{y_p = Ae^{-2x} \cos x + Be^{-2x} \sin x}$

\subsection*{3.6 Variation of Parameters}
\dfn{Variation of Parameters}{
  To solve the ODE
  $$ y^{\prime \prime}+P(t) y^{\prime}+Q(t) y=f(t), $$
  we find the complementary solution to the corresponding homogeneous ODE, $y^{\prime \prime}+P(t) y^{\prime}+Q(t) y=0$, and then find the particular solution of the form
  $$ y_p=u_1 y_1+u_2 y_2 $$where $u_1^{\prime}=-\frac{y_2 f(t)}{W}$ and $u_2^{\prime}=\frac{y_1 f(t)}{W}$, with $\mathrm{W}$ being the Wronskian.
  Therefore, the general solution is $y=y_c+y_p$.
}

\qs{}{
  Solve
  $$ y^{\prime \prime}-4 y^{\prime}+4 y=(t+1) e^{2 t} $$
}
\sol The auxiliary equation is $r^2 - 4r + 4 = 0 \implies (r-2)^2 = 0 \implies r = 2$.
Such that $y_c = c_1 e^{2t} + c_2 t e^{2t}$. This means that $y_1 = e^{2t}$ and $y_2 = t e^{2t}$.
\double
We can find the Wronskian such that $W = \left| \begin{array}{cc} e^{2t} & t e^{2t} \\ 2e^{2t} & (1+2t)e^{2t} \end{array} \right| = e^{4t}$.
\double
Where 
\begin{align*}
u_1^{\prime} &= -\frac{te^{2t}(t+1)e^{2t}}{e^{4t}} = -t^2-t \implies u_1 = -\frac{t^3}{3} - \frac{t^2}{2} \\
u_2^{\prime} &= \frac{e^{2t}(t+1)e^{2t}}{e^{4t}} = t+1 \implies u_2 = \frac{t^2}{2} + t \\
y_p &= \left(-\frac{t^3}{3} - \frac{t^2}{2}\right) e^{2t} + \left(\frac{t^2}{2} + t\right) t e^{2t} \\
y_p &= \frac{1}{6}t^3 e^{2t} + \frac{1}{2}t^2 e^{2t}
\end{align*}
Therefore the general solution is
$$\boxed{y = c_1 e^{2t} + c_2 t e^{2t} + \frac{1}{6}t^3 e^{2t} + \frac{1}{2}t^2 e^{2t}}$$

\qs{}{
Solve  
$$4 y^{\prime \prime}+36 y=\csc 3 t $$
}
\sol To use variations of parameters we must first have $y''$ as the leading term. Therefore we divide by 4 such that $y'' + 9y = \frac{1}{4} \csc 3t$.
\double
The auxiliary equation is $r^2 + 9 = 0 \implies r = 0 \pm 3i$. Such that $y_c = c_1 \cos 3t + c_2 \sin 3t$.
Where $y_1 = \cos 3t$ and $y_2 = \sin 3t$.
\double
We can find the Wronskian such that $W = \left| \begin{array}{cc} \cos 3t & \sin 3t \\ -3\sin 3t & 3\cos 3t \end{array} \right| = 3$.
\double
Where
\begin{align*}
u_1^{\prime} &= -\frac{\sin 3t \frac{1}{4 \sin 3t}}{3} = -\frac{1}{12} \implies u_1 = -\frac{1}{12} t \\
u_2^{\prime} &= \frac{\cos 3t \frac{1}{4\sin 3t}}{3} = \frac{1}{12} \cot 3t \implies u_2 = \frac{1}{36} \ln |\sin 3t| \\
y_p &= -\frac{1}{12} t \cos 3t + \frac{1}{36} \ln |\sin 3t| \sin 3t
\end{align*}
Therefore the general solution is
$$\boxed{y = c_1 \cos 3t + c_2 \sin 3t -\frac{1}{12} t \cos 3t + \frac{1}{36} \sin 3t \ln |\sin 3t|}$$

\subsection*{3.7 Mechanical and Electrical Vibrations}
\imgg{3.7}{0.7}
Given that $F=ma$ we can say that $a = u''$. We can also say that in a static case we have that $mg - kL = 0 \implies mg = kL$.
Where if we have some damping force $\gamma$ and external force $F(t)$, we can say that
$$ m u'' + \gamma u' + ku = F(t) $$

\qs{}{
  A mass weighing $4 \mathrm{lb}$ stretches a spring $2 \mathrm{in}$. Suppose that the mass is given an additional 6-in displacement in the positive direction and then released. The mass is in a medium that exerts a viscous resistance of $6 \mathrm{lb}$ when the mass has a velocity of $3 \mathrm{ft} / \mathrm{s}$. Under the assumptions discussed in this section, formulate the initial value problem that governs the motion of the mass.
}
\sol We are given a weight of 4 lbs but we want mass, such that $m = \frac{W}{g} = \frac{4}{32} = \frac{1}{8} \frac{\text{lb} \cdot \text{s}^2}{\text{ft}}$.
\double
We are given a viscous resistance of 6 lbs, in other words $F_\gamma = 6$. Since $6 = \gamma u' = \gamma \cdot 3 \implies \gamma = 2$.
\double
We are given a displacement of 2 inches ($\frac{1}{6}$ feet), such that by hook's law, we have, $F = kx \implies 4 = k \cdot \frac{1}{6} \implies k = 24$.
\double
Setting up the full equation of motion, we have
$$ \frac{1}{8} u'' + 2u' + 24u = 0 $$
Since we know that at $t=0$ the mass is given an additional 6-in ($\frac{1}{2}$ feet) displacement in the positive direction and then released, we have that
$ u(0) = \frac{1}{2}$ and $u'(0) = 0$.
\begin{align*}
  u'' + 16u' + 192u = 0 &\implies 1r^2 + 16r +192 = 0
  \intertext{Doing the quadratic formula (im not typing that out lol)...}
  r &= -8 \pm 8i\sqrt{2} 
  \intertext{Such that $\alpha = -8$ and $\beta = 8\sqrt{2}$}
  u &= e^{-8t} (c_1 \cos 8\sqrt{2}t + c_2 \sin 8\sqrt{2}t)
  \intertext{Given that $u(0) = \frac{1}{2}$}
  \frac{1}{2} &= c_1 \cos 0 + c_2 \sin 0 = c_1 \implies c_1 = \frac{1}{2}
  \intertext{Given that $u'(0) = 0$}
  u' &= -8 e^{-8t} ( \frac{1}{2} \cos (3 \sqrt{2} t ) + c_2 \sin (8 \sqrt{2} t)) + e^{-8t} (-4 \sqrt{2} \sin (8 \sqrt{2} t ) + 8 \sqrt{2} c_2 \cos (8 \sqrt{2} t)) \\
  0 &= -8 (\frac{1}{2}) + 8 \sqrt{2} c_2 \implies 4 = 8\sqrt{2} c_2 \implies c_2 = \frac{\sqrt{2}}{4}
  \intertext{Therefore the general solution is}
  \Aboxed{u &= e^{-8t} \left( \frac{1}{2} \cos (8 \sqrt{2} t ) + \frac{\sqrt{2}}{4} \sin (8 \sqrt{2} t) \right)}
\end{align*}

\subsection*{6.1}

\dfn{Laplace Transform}{
  Let $\mathrm{f}$ be a function defined for $t \geq 0$. Then the integral
  $$ L\{\mathrm{f}(\mathrm{t})\}=\int_0^{\infty} e^{-s t} f(t) d t $$
  is said to be the Laplace Transform of $\mathrm{f}$, provided that the integral converges.
}

\qs{}{
  Evaluate $L\{1\}$
}
\sol 
\begin{align*}
  \int_0^{\infty} e^{-st} (1) dt &= \lim_{b \to \infty} \int_0^b e^{-st} dt \\
  &= \lim_{b \to \infty} \left[ \frac{e^{-st}}{-s} \right]_0^b = \boxed{\frac{1}{s}}
\end{align*}

\qs{}{
  Evaluate $L\{t\}$
}
\sol
\begin{align*}
  \int_0^{\infty} e^{-st} (t) dt &= \lim_{b \to \infty} \int_0^b te^{-st} dt \\
  \intertext{Using integration by parts, let $u = t$ and $dv = e^{-st} dt$ such that $du = dt$ and $v = \frac{e^{-st}}{-s}$}
  &= -\frac{t}{se^{st}} + \frac{1}{s} \int_0^b e^{-st} dt \\
  &= \lim_{b \to \infty} \left[-\frac{t}{se^{st}} - \frac{1}{s^2} e^{-st} \right]_0^b \\
  &= 0 - 0 - (0 - \frac{1}{s^2}) = \boxed{\frac{1}{s^2}}
\end{align*}

\qs{}{
  Evaluate $L\{e^{-3t}\}$
}
\sol
\begin{align*}
  \int_0^{\infty} e^{-(s+3)t} dt &= -\frac{1}{(s+3)e^{(s+3)t}} \Big|_0^{\infty} \\
  &= 0 + \frac{1}{(s+3)} = \boxed{\frac{1}{s+3}}
\end{align*}

\thm{}{
  Given $L\{e^{at}\}$ such that $s > 0$, we have that
  $$L\{e^{at}\} = \frac{1}{s-a}$$
}

\qs{}{
  Evaluate $\mathrm{L}\{\sin 2 \mathrm{t}\}$
}
\sol Skipping the work (integration by parts), we have that $\boxed{\frac{2}{s^2 + 4}}$.

\thm{}{
  Given $L\{\sin \mathrm{bt}\}$ such that $s > 0$, we have that
  $$L\{\sin \mathrm{bt}\} = \frac{b}{s^2 + b^2}$$
}

\dfn{Laplace Transform is a Linear Operator}{
  Given generic functions $f(t)$ and $g(t)$ and constants $a$ and $b$, we have that
  $$L\{a f(t)+b g(t)\}=a L\{f(t)\}+b L\{g(t)\}$$
}

\qs{}{
  Use the Laplace transform formulas to find the following Laplace transforms: \\
  a) $L\{1+5 t\}$ \\
  b) $\mathrm{L}\left\{4 e^{5 t}-10 \sin 2 t\right\}$
}
\sol
a) $L\{1+5t\} = L\{1\} + 5L\{t\} = \boxed{\frac{1}{s} + \frac{5}{s^2}}$
\double
b) $L\{4e^{5t} - 10 \sin 2t\} = 4L\{e^{5t}\} - 10L\{\sin 2t\} = 4 \left(\frac{1}{s-5}\right) - 10 \left(\frac{2}{s^2 + 4}\right) = \boxed{\frac{4}{s-5} - \frac{20}{s^2 + 4}}$

\qs{}{
  Use the definition of the Laplace transform to find $L\{f(t)\}$ where
$$
f(t)=\left\{\begin{array}{lr}
0, & 0 \leq t<3 \\
2, & t \geq 3
\end{array}\right.
$$}
\sol
\begin{align*}
  \int_0^{\infty} e^{-st} f(t) dt &= \int_0^3 e^{-st} (0) dt + \int_3^{\infty} e^{-st} (2) dt \\
  &= 0 + - \frac{2}{s} e^{-st} \Big|_3^{\infty} = 0 + \frac{2}{s} e^{-3s} = \boxed{\frac{2}{s} e^{-3s}}
\end{align*}

\dfn{Laplace Transform of Derivative(s)}{
  Given $L\{f(t)\} = F(s)$, we have that
  $$L\{f^{\prime}(t)\} = sF(s) - f(0)$$
  $$L\{f^{\prime \prime}(t)\} = s^2 F(s) - sf(0) - f^{\prime}(0)$$
}

\subsection*{6.2}

\dfn{}{
  If $\mathrm{L}\{\mathrm{f}(\mathrm{t})\}=\mathrm{F}(\mathrm{s})$, then
  $$\mathrm{L}\left\{e^{a t} f(t)\right\}=\mathrm{F}(\mathrm{s}-\mathrm{a})$$
}

\qs{}{
  Evaluate \\
  a) $\mathrm{L}\left\{e^{5 t} t^3\right\}$ \\
  b) $\mathrm{L}\left\{e^{-2 t} \cos 4 t\right\}$
}
\sol \\
a) 
\begin{align*}
  L\{t^3\} &= \frac{3!}{s^{3+1}} = \frac{6}{s^4} \\
  &\implies \boxed{\frac{6}{(s-5)^4}}
\end{align*}
\noindent
b)
\begin{align*}
  L\{\cos 4t\} &= \frac{s}{s^2 + 16} \\
  &\implies \boxed{\frac{s+2}{(s+2)^2 + 16}}
\end{align*}

\dfn{Inverse Laplace Transform}{
  If $\mathrm{F}(\mathrm{s})$ represents the Laplace Transform of $f(t)$,
  we say $f(t)$ is the inverse Laplace Transform of $F(s)$ and we write
  $$f(t)=L^{-1}\{F(s)\}$$
}

\qs{}{
  Evaluate a) $L^{-1}\left\{\frac{1}{s^5}\right\}$ \\
  b) $L^{-1}\left\{\frac{1}{s^2+7}\right\}$
}
\sol \\
a) We know that $L\{t^n\} = \dfrac{n!}{s^{n+1}}$
such that we can rewrite $\dfrac{1}{4!} \cdot L\{\dfrac{4!}{s^{4+1}}\}$,
therefore $L^{-1}\{\dfrac{1}{s^5}\} = \boxed{\dfrac{t^4}{24}}$
\double
b) We know that $L\{ \sin bt \} = \dfrac{b}{s^2 + b^2}$,
such that we can rewrite $\dfrac{1}{\sqrt{7}} \cdot L\{ \dfrac{\sqrt{7}}{s^2 + 7} \}$,
therefore $L^{-1}\{ \dfrac{1}{s^2 + 7} \} = \boxed{\dfrac{1}{\sqrt{7}} \sin \sqrt{7}t}$

\dfn{Inverse Laplace Transform is a Linear Operator}{
  Given $F(s)$ and $G(s)$, with constants $a$ and $b$, we have that
  $$\mathrm{L}^{-1}\{\mathrm{aF}(\mathrm{s})+\mathrm{bG}(\mathrm{s})\}=\mathrm{af}(\mathrm{t})+\mathrm{bg}(\mathrm{t})$$
}

\qs{}{
  Evaluate \\
  a) $$\mathrm{L}^{-1}\left\{\frac{-2 s+6}{s^2+4}\right\}$$
  b) $$\mathrm{L}^{-1}\left\{\frac{s^2+6 s+9}{(s-1)(s-2)(s+4)}\right\}$$
}
\sol a) We have both $s$ and $b$ in the numerator such that we can rewrite the expression as
\begin{align*}
  L^{-1}\left\{\frac{-2s+6}{s^2+4}\right\} &= -2L^{-1}\left\{\frac{s}{s^2+4}\right\} + 6L^{-1}\left\{\frac{1}{s^2+4}\right\} \\
  &=  \boxed{-2\cos(2t) + 3\sin(2t)}
\end{align*}
\double
b)
  $$
  \dfrac{s^2+6s+9}{(s-1)(s-2)(s+4)} = \frac{A}{s-1} + \frac{B}{s-2} + \frac{C}{s+4} 
  $$
Using the coverup method...
$$
  A = \frac{1^2+6(1)+9}{(1-2)(1+4)} = -\frac{16}{5}, \quad B = \frac{25}{6} \quad C = \frac{1}{30}
$$
We get
$$ -\frac{16}{5} L^{-1}\{ \frac{1}{s-1} \} + \frac{25}{6} L^{-1}\{ \frac{1}{s-2} \} + \frac{1}{30} L^{-1}\{ \frac{1}{s+4} \} = \boxed{-\frac{16}{5} e^t + \frac{25}{6} e^{2t} + \frac{1}{30} e^{-4t}} $$

\qs{}{
  Evaluate
  $$L^{-1}\left\{\frac{2 s+5}{(s-3)^2}\right\}$$
}
\sol We know that $L\{t^n e^{at}\} = \dfrac{n!}{s^{n+1}}$ such that we can rewrite the expression via partial fractions.
\begin{align*}
  \dfrac{2s+5}{(s-3)^2} &= \dfrac{A}{s-3} + \dfrac{B}{(s-3)^2} \\
  2s+5 &= A(s-3) + B
  \implies A = 2, \quad B = 11
\end{align*}
Such that
$$ 2L^{-1}\left\{\frac{1}{s-3}\right\} + 11L^{-1}\left\{\frac{1}{(s-3)^2}\right\} = \boxed{2e^{3t} + 11te^{3t}} $$

\thm{}{
  $$\mathrm{L}^{-1}\left\{\frac{A(s-a)+B b}{(s-a)^2+b^2}\right\}=\mathrm{e}^{\mathrm{at}}(\mathrm{A} \cos (\mathrm{bt})+\mathrm{B} \sin (\mathrm{bt}))$$
}

\qs{}{
  Evaluate
  $$L^{-1}\left\{\frac{s / 2+5 / 3}{s^2+4 s+6}\right\}$$
}
\sol Notice that we can't factor the denominator, such that we must complete the square.
$$ L^{-1}\left\{\frac{s / 2+5 / 3}{s^2+4 s+6}\right\} = L^{-1}\left\{\frac{s/2 + 5/3}{(s+2)^2 + 2}\right\} $$
Such that $a = -2$ and $b = \sqrt{2}$. where we can orient the expression to fit the form of the theorem.
$$ L^{-1}\left\{\frac{\frac{1}{2}(s+2) + 2/3}{(s+2)^2 + 2}\right\}$$
We know that $B = \frac{\sqrt{2}}{3}$ and $A = \frac{1}{2}$ such that
$$ \boxed{e^{-2t}\left(\frac{1}{2} \cos (\sqrt{2}t) + \frac{\sqrt{2}}{3} \sin (\sqrt{2}t)\right)} $$

\qs{}{
  Use Laplace transforms to solve the initial-value problem
  $$\frac{d y}{d t}+3 y=13 \sin 2 t, \quad \mathrm{y}(0)=6$$
}
\sol
\begin{align*}
  L\{\frac{dy}{dt}\} + 3L\{y\} &= 13 L\{\sin 2t\} \\
  sY(s) - y(0) + 3 Y(s) &= 13 \left(\frac{2}{s^2 + 4}\right) \\
  sY(s) - 6 + 3Y(s) &= \frac{26}{s^2 + 4} \\
  (s+3)Y(s) &= 6 + \frac{26}{s^2 + 4} \\
  Y(s) &= \frac{6}{s+3} + \frac{26}{(s+3)(s^2 + 4)} \\
  &= \frac{6(s^2+4)+26}{(s+3)(s^2 +4)} \\
  Y(s) &= \frac{6s^2+50}{(s+3)(s^2+4)}
\end{align*}
Using partial fractions, we get
\begin{align*}
  \dfrac{6s^2+50}{(s+3)(s^2+4)} &= \frac{A}{(s+3)} + \frac{Bs+C}{s^2+4} \\
  6s^2 + 50 &= A(s^2+4) + (Bs+C)(s+3) \\
  6s^2 + 50 &= As^2 + 4A + Bs^2 + 3Bs + Cs + 3C
\end{align*}
Equating coefficients we get $A = 8$, $B = -2$, and $C = 6$, therefore
\begin{align*}
  y(t) &= L^{-1} \left\{ \frac{8}{s+3} + \frac{-2s+6}{s^2+4} \right\}
  \intertext{Refering to question 2.39.a}
  &= \boxed{8e^{-3t} - 2\cos 2t + 3\sin 2t}
\end{align*}

\qs{}{
  Use Laplace transforms to solve the initial-value problem
  $$y^{\prime \prime}-3 y^{\prime}+2 y=e^{-4 t}, \quad \mathrm{y}(0)=1, \quad \mathrm{y}^{\prime}(0)=5$$
}
\sol 
\begin{align*}
  L\{ y'' \} - 3L\{ y' \} + 2L\{ y \} &= L\{ e^{-4t} \} \\
  s^2 Y(s) -s y(0) - y'(0) - 3(sY(s) - y(0)) + 2Y(s) &= \frac{1}{s+4} \\
  s^2 Y(s) -s -2 - 3sY(s) + 2Y(s) &= \frac{1}{s+4} \\
  (s^2-3s+2) Y(s) &= s + 2 + \frac{1}{s+4}
\end{align*}
Notice that $s^2-3s+2$ is our auxiliary polynomial
\begin{align*}
  Y(s) &= \dfrac{s+2}{(s-1)(s-2)} + \dfrac{1}{(s+4)(s-1)(s-2)} \\
  &= \frac{s^2 + 6s + 9}{(s-1)(s-2)(s+4)}
  \intertext{Refering to question 2.39.b}
  \Aboxed{y(t) &= -\frac{16}{5} e^t + \frac{25}{6} e^{2t} + \frac{1}{30} e^{-4t}}
\end{align*}

\qs{}{
  Use Laplace transforms to solve the initial-value problem
  $$y^{\prime \prime}-6 y^{\prime}+9 y=t^2 e^{3 t}, \quad \mathrm{y}(0)=2, \quad \mathrm{y}^{\prime}(0)=17$$
}
\sol
\begin{align*}
  L\{y''\} - 6L\{y'\} + 9L\{y\} &= L\{t^2 e^{3t}\} \\
  s^2 Y(s) - s y(0) - y'(0) - 6(sY(s) - y(0)) + 9Y(s) &= \frac{2}{(s-3)^3} \\
  s^2 Y(s) - 2s - 17 - 6s Y(s) + 12 + 9 Y(s) &= \frac{2}{(s-3)^3} \\
  Y(s)(s^2 - 6s + 9) &= 2s + 5 + \frac{2}{(s-3)^3}
\end{align*}
\begin{align*}
  Y(s) &= \underbrace{\frac{2s+5}{(s-3)^2}}_{\text{Performing PFD}} + \frac{2}{(s-3)^5} \\
  &= \frac{2}{s-3} + \frac{11}{(s-3)^2} + \frac{2}{(s-3)^5} \\
  y(t) &= 2 L^{-1}\left\{\frac{1}{s-3}\right\} + 11 L^{-1}\left\{\frac{1}{(s-3)^2}\right\} + 2 L^{-1}\left\{\frac{1}{(s-3)^5}\right\} \\
  \Aboxed{y(t) &= 2 e^{3t} + 11te^{3t} + \frac{1}{12} t^4 e^{3t}}
\end{align*}

\qs{}{
  Use Laplace transforms to solve the initial-value problem
  $$y^{\prime \prime}+4 y^{\prime}+6 y=1+e^{-t}, \quad \mathrm{y}(0)=0, \quad \mathrm{y}^{\prime}(0)=0$$
}
\sol
\begin{align*}
  L \{ y'' \} + 4L\{y'\} + 6L\{y\} &= L\{1\} + L\{e^{-t}\} \\
  s^2 Y(s) - sy(0) - y'(0) + 4\left(sY(s) - y(0)\right) + 6Y(s) &= \frac{1}{s} + \frac{1}{s+1} \\
  Y(s) (s^2 + 4s + 6) &= \frac{1}{s} + \frac{1}{s+1} \\
  Y(s) (s^2 + 4s + 6) &= \frac{2s+1}{s(s+1)}
\end{align*}
$$ y = L^{-1}\left\{\frac{2s+1}{s(s+1)(s^2+4s+6)}\right\} $$
Using partial fractions,
\begin{align*}
  \frac{2s+1}{s (s+1) (s^2+4s+6)} &= \frac{A}{s} + \frac{B}{s+1} + \frac{Cs+D}{s^2+4s+6} \\
  2s+1 &= A(s+1)(s^2+4s+6) + B(s)(s^2+4s+6) + (Cs+D)(s)(s+1) \\
  2s+1 &= As^3 + 5 As^2 + 10 As + 6A + Bs^3 + 4Bs^2 + 6Bs + Cs^3 + Cs^2 + Ds^2 + Ds \\
  s^3 &: A + B + C = 0 \\
  s^2 &: 5A + 4B + C + D = 0 \\
  s &: 10A + 6B + D = 2 \\
  \# &: 6A = 1
\end{align*}
RREF a matrix, we get
$$A = \frac{1}{6}, \quad B = \frac{1}{3}, \quad C = -\frac{1}{2}, \quad D = -\frac{5}{3}$$
\begin{align*}
  y &= \frac{1}{6} L^{-1}\left\{\frac{1}{s}\right\} + \frac{1}{3} L^{-1}\left\{\frac{1}{s+1}\right\} - L^{-1} \left\{\frac{\frac{1}{2}s + \frac{5}{3}}{s^2+4s+6}\right\} \\
  &= \frac{1}{6} + \frac{1}{3} e^{-t} - \underbrace{L^{-1} \left\{\frac{\frac{1}{2}s + \frac{5}{3}}{s^2+4s+6}\right\}}_{\text{Refering to qs 2.41}} \\
  y &= \frac{1}{6} + \frac{1}{3} e^{-t} - \left( \frac{1}{2} e^{-2t} \cos (\sqrt{2}t) + \frac{\sqrt{2}}{3} e^{-2t} \sin(\sqrt{2}t) \right) \\
  \Aboxed{y &= \frac{1}{6} + \frac{1}{3} e^{-t} - \frac{1}{2} e^{-2t} \cos (\sqrt{2}t) - \frac{\sqrt{2}}{3} e^{-2t} \sin(\sqrt{2}t)}
\end{align*}

\dfn{Laplace of Higher Derivatives}{
  Given $L\{f(t)\} = F(s)$, we have that
  $$L\{f^{(3)}(t)\} = s^3 F(s) - s^2 f(0) - sf'(0) - f''(0)$$
  $$L\{f^{(4)}(t)\} = s^4 F(s) - s^3 f(0) - s^2 f'(0) - sf''(0) - f'''(0)$$
}

\subsection*{6.3 \& 6.4}

\dfn{Unit Step Function}{
  To deal effectively with functions having jump discontinuities (binary states),
  $$\mathrm{u}_{\mathrm{a}}(\mathrm{t})=\left\{\begin{array}{l}0,0 \leq t<a \\ 1, t \geq a\end{array}\right.$$
}

\dfn{Rewriting a Piecewise Funciton as a Unit Step Function}{
  The piecewise-defined function $f(t)=\left\{\begin{array}{ll}g(t), & t<a \\ h(t), & t \geq a\end{array}\right.$
  can be rewritten using unit step functions as
  $$f(t)=g(t)-g(t) u_a(t)+h(t) u_a(t)$$
  
  A function of the type $f(t)=\left\{\begin{array}{cc}0, & t<a \\ g(t), & a \leq t<b \\ 0, & t \geq b\end{array} \quad\right.$
  can be written in terms of unit step functions as
  $$f(t)=g(t)\left[u_a(t)-u_b(t)\right]$$
}

\qs{}{
  Express $f(t)=\left\{\begin{array}{cc}20 t, & t<5 \\ 0, & t \geq 5\end{array}\right.$ in terms of unit step functions.
}
\sol Given that $g(t) = 20t$ and $h(t) = 0$,
\begin{align*}
  f(t) &= 20t - 20t u_5(t) \\
  &= \boxed{20t - 20t u_5(t)}
\end{align*}

\dfn{Shifted Unit Step Function}{
  Written as a piecewise-defined function,
$$
\mathrm{f}(\mathrm{t}-\mathrm{a}) \mathrm{u}_{\mathrm{a}}(\mathrm{t})=\left\{\begin{array}{r}
0,0 \leq t<a \\
f(t-a), t \geq a
\end{array}\right.
$$
}
\thm{Laplace Transform of a Unit Step Function}{
  $$\mathrm{L}\left\{\mathrm{u}_{\mathrm{a}}(\mathrm{t})\right\}=\frac{e^{-a s}}{s}$$
}
\thm{Laplace Transform of a Shifted Unit Step Function}{
  $$\mathrm{L}\left\{\mathrm{f}(\mathrm{t}-\mathrm{a}) \mathrm{u}_{\mathrm{a}}(\mathrm{t})\right\}=e^{-a s} F(s)$$
}

\qs{}{
  Find the Laplace transform of
  $$\mathrm{f}(\mathrm{t})=2-3 \mathrm{u}_2(\mathrm{t})+\mathrm{u}_3(\mathrm{t})$$
}
\sol
\begin{align*}
  L \{f(t)\} &= L\{2\} - 3L\{u_2(t)\} + L\{u_3(t)\} \\
  &= \boxed{\frac{2}{s} - \frac{3e^{-2s}}{s} + \frac{e^{-3s}}{s}}
\end{align*}

\thm{Laplace Transform (Sit. Trig)}{
  $$\mathrm{L}\left\{\mathrm{g}(\mathrm{t}) \mathrm{u}_{\mathrm{a}}(\mathrm{t})\right\}=e^{-a s} \mathrm{~L}\{g(t+a)\}$$
}

\qs{}{
  Evaluate $$\mathrm{L}\left\{\cos (\mathrm{t}) u_\pi(t)\right\}$$
}
\sol Referring to thm 2.9,
\begin{align*}
  \mathrm{L}\left\{\cos (\mathrm{t}) u_\pi(t)\right\} &= e^{-as} L \left\{ \cos (t+\pi) \right\} \\
  &= e^{-\pi s} L \left\{ \cos t \cos \pi - \sin t \sin \pi \right\} \\
  &= e^{-\pi s} L \left\{ -\cos t \right\} \\
  &= -e^{-\pi s} \left(\frac{s}{s^2+1}\right) \\
  &= \boxed{-\frac{s e^{-\pi s}}{s^2+1}}
\end{align*}

\qs{}{
  Evaluate
  $$\mathrm{L}^{-1}\left\{\frac{e^{-2 s}}{s-4}\right\}$$
}
\sol
\begin{align*}
  & \mathrm{L}^{-1}\left\{e^{-2 s} \cdot \underbrace{\frac{1}{s-4}}_{e^{4t}}\right\}
  \intertext{Referring to thm 2.9}
  &= \boxed{e^{4(t-2) u_2(t)}}
\end{align*}

\qs{}{
  Evaluate
  $$\mathrm{L}^{-1}\left\{\frac{e^{-\pi s / 2} s}{s^2+9}\right\}$$
}
\begin{align*}
  & \mathrm{L}^{-1}\left\{e^{-\pi s / 2} \cdot \underbrace{\frac{s}{s^2+9}}_{\cos (3t)} \right\} \\
  &= \boxed{\cos 3 (t- \frac{\pi}{2}) u_{\frac{\pi}{2}}(t)}
\end{align*}

\qs{}{
  Solve $y^{\prime}+y=f(t),\quad y(0)=5$, where $f(t)=\left\{\begin{array}{cc}0 . & t<\pi \\ 3 \cos t, & t \geq \pi\end{array}\right.$
}

\begin{align*}
  y' + y &= \left\{\begin{array}{cc}0 . & t<\pi \\ 3 \cos t, & t \geq \pi\end{array}\right. \\
  &= 3 \cos t u_{\pi}(t) \\
  L\{y'\} + L\{y\} &= \underbrace{3L\{\cos t u_{\pi}(t)\}}_{\text{Solved in qs 2.48}} \\
  sY(s) - y(0) + Y(s) &= -\frac{se^{-\pi s}}{s^2 + 1} \\
  Y(s) (s+1)& = 5 - 3\frac{se^{-\pi s}}{s^2 + 1} \\
  Y(s) &= \frac{5}{s+1} - 3 e^{-\pi s} \cdot \frac{s}{(s+1)(s^2 + 1)}
\end{align*}
We need to break down the last term using partial fractions.
\begin{align*}
  \frac{s}{(s+1)(s^2 + 1)} &= \frac{A}{s+1} + \frac{Bs+C}{s^2+1} \\
  s &= A(s^2+1) + (Bs+C)(s+1) \\
  s &= As^2 + A + Bs^2 + Bs + Cs + C \\
  \implies & A = -\frac{1}{2}, \quad B = \frac{1}{2}, \quad C = \frac{1}{2}
\end{align*}
Such that,
\begin{align*}
  y &= 5 L^{-1} \left\{\frac{1}{s+1}\right\} - 3 L^{-1} \left\{ e^{-\pi s} \left(\frac{-\frac{1}{2}}{(s+1)} + \frac{\frac{1}{2}s + \frac{1}{2}}{s^2+1} \right) \right\} \\
  &= 5 L^{-1} \left\{\frac{1}{s+1}\right\} + \frac{3}{2} L^{-1} \left\{ e^{-\pi s} \left(\frac{1}{(s+1)} - \frac{s + 1}{s^2+1} \right) \right\} \\
  &= 5 e^{-t} + \frac{3}{2} L^{-1} \left\{ e^{-\pi s} \left( \underbrace{\frac{1}{s+1}}_{e^{-t}} - \underbrace{\frac{s}{s^2+1}}_{\cos t} - \underbrace{\frac{1}{s^2 +1}}_{\sin t} \right) \right\} \\
  \Aboxed{y &= 5 e^{-t} + \frac{3}{2} \left( e^{-(t-\pi)} - \cos(t-\pi) - \sin(t-\pi) \right) u_{\pi} (t)}
\end{align*}

\thm{Laplace Periodic Function}{
  If $f(t)$ is a periodic function of period $T$, then
$$ \mathrm{L}\{\mathrm{f}(\mathrm{t})\}=\frac{1}{1-e^{-s T}} \int_0^T e^{-s t} f(t) d t . $$
}

\subsection*{6.5}
\dfn{Shifted Dirac Delta Function}{
  The shifted Dirac delta function is defined as
  $$\delta(t-a)=\left\{\begin{array}{ll}0, & t \neq a \\ \infty, & t=a\end{array}\right.$$
  such that
  $$\int_{-\infty}^{\infty} f(t) \delta(t-a) d t=f(a)$$
}
\thm{Laplace Transform of Dirac Delta Function}{
  $$\mathrm{L}\{\delta(t-a)\}=e^{-a s}$$
}

\qs{}{
  Solve $y^{\prime \prime}+y=4 \delta(t-2 \pi)$, where $y(0)=1, \quad y^{\prime}(0)=0$
}
\sol
\begin{gather*}
  L\{y''\} + L\{y\} = 4L\{\delta(t-2\pi)\} \\
  s^2 Y(s) - \underbrace{sy(0)}_{s} - \underbrace{y'(0)}_{0} + Y(s) = 4e^{-2\pi s} \\
  Y(s)(s^2 + 1) = s + 4e^{-2\pi s} \\
  Y(s) = \frac{s}{s^2 + 1} + 4e^{-2\pi s} \cdot \frac{1}{s^2 + 1} \\
  y(t) = L^{-1}\left\{\frac{s}{s^2 + 1}\right\} + 4L^{-1}\left\{e^{-2\pi s} \cdot \underbrace{\frac{1}{s^2 + 1}}_{\sin t}\right\} \\
  \boxed{y(t) = \cos t + 4 \sin (t - 2\pi) u_{2\pi}(t)}
  \intertext{We can rewrite this as a piecewise function,}
  \boxed{y(t) = \left\{\begin{array}{cc}\cos t, & t<2\pi \\ \cos t + 4 \sin t, & t \geq 2\pi\end{array}\right.}
\end{gather*}
If we were to change the initial conditions to $y(0) = 0$ and $y'(0) = 0$, we would get
$$ y = 4 \sin (t - 2\pi) u_{2\pi}(t) $$

\end{document}